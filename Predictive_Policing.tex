% Options for packages loaded elsewhere
\PassOptionsToPackage{unicode}{hyperref}
\PassOptionsToPackage{hyphens}{url}
%
\documentclass[
]{article}
\usepackage{amsmath,amssymb}
\usepackage{iftex}
\ifPDFTeX
  \usepackage[T1]{fontenc}
  \usepackage[utf8]{inputenc}
  \usepackage{textcomp} % provide euro and other symbols
\else % if luatex or xetex
  \usepackage{unicode-math} % this also loads fontspec
  \defaultfontfeatures{Scale=MatchLowercase}
  \defaultfontfeatures[\rmfamily]{Ligatures=TeX,Scale=1}
\fi
\usepackage{lmodern}
\ifPDFTeX\else
  % xetex/luatex font selection
\fi
% Use upquote if available, for straight quotes in verbatim environments
\IfFileExists{upquote.sty}{\usepackage{upquote}}{}
\IfFileExists{microtype.sty}{% use microtype if available
  \usepackage[]{microtype}
  \UseMicrotypeSet[protrusion]{basicmath} % disable protrusion for tt fonts
}{}
\makeatletter
\@ifundefined{KOMAClassName}{% if non-KOMA class
  \IfFileExists{parskip.sty}{%
    \usepackage{parskip}
  }{% else
    \setlength{\parindent}{0pt}
    \setlength{\parskip}{6pt plus 2pt minus 1pt}}
}{% if KOMA class
  \KOMAoptions{parskip=half}}
\makeatother
\usepackage{xcolor}
\usepackage[margin=1in]{geometry}
\usepackage{color}
\usepackage{fancyvrb}
\newcommand{\VerbBar}{|}
\newcommand{\VERB}{\Verb[commandchars=\\\{\}]}
\DefineVerbatimEnvironment{Highlighting}{Verbatim}{commandchars=\\\{\}}
% Add ',fontsize=\small' for more characters per line
\usepackage{framed}
\definecolor{shadecolor}{RGB}{248,248,248}
\newenvironment{Shaded}{\begin{snugshade}}{\end{snugshade}}
\newcommand{\AlertTok}[1]{\textcolor[rgb]{0.94,0.16,0.16}{#1}}
\newcommand{\AnnotationTok}[1]{\textcolor[rgb]{0.56,0.35,0.01}{\textbf{\textit{#1}}}}
\newcommand{\AttributeTok}[1]{\textcolor[rgb]{0.13,0.29,0.53}{#1}}
\newcommand{\BaseNTok}[1]{\textcolor[rgb]{0.00,0.00,0.81}{#1}}
\newcommand{\BuiltInTok}[1]{#1}
\newcommand{\CharTok}[1]{\textcolor[rgb]{0.31,0.60,0.02}{#1}}
\newcommand{\CommentTok}[1]{\textcolor[rgb]{0.56,0.35,0.01}{\textit{#1}}}
\newcommand{\CommentVarTok}[1]{\textcolor[rgb]{0.56,0.35,0.01}{\textbf{\textit{#1}}}}
\newcommand{\ConstantTok}[1]{\textcolor[rgb]{0.56,0.35,0.01}{#1}}
\newcommand{\ControlFlowTok}[1]{\textcolor[rgb]{0.13,0.29,0.53}{\textbf{#1}}}
\newcommand{\DataTypeTok}[1]{\textcolor[rgb]{0.13,0.29,0.53}{#1}}
\newcommand{\DecValTok}[1]{\textcolor[rgb]{0.00,0.00,0.81}{#1}}
\newcommand{\DocumentationTok}[1]{\textcolor[rgb]{0.56,0.35,0.01}{\textbf{\textit{#1}}}}
\newcommand{\ErrorTok}[1]{\textcolor[rgb]{0.64,0.00,0.00}{\textbf{#1}}}
\newcommand{\ExtensionTok}[1]{#1}
\newcommand{\FloatTok}[1]{\textcolor[rgb]{0.00,0.00,0.81}{#1}}
\newcommand{\FunctionTok}[1]{\textcolor[rgb]{0.13,0.29,0.53}{\textbf{#1}}}
\newcommand{\ImportTok}[1]{#1}
\newcommand{\InformationTok}[1]{\textcolor[rgb]{0.56,0.35,0.01}{\textbf{\textit{#1}}}}
\newcommand{\KeywordTok}[1]{\textcolor[rgb]{0.13,0.29,0.53}{\textbf{#1}}}
\newcommand{\NormalTok}[1]{#1}
\newcommand{\OperatorTok}[1]{\textcolor[rgb]{0.81,0.36,0.00}{\textbf{#1}}}
\newcommand{\OtherTok}[1]{\textcolor[rgb]{0.56,0.35,0.01}{#1}}
\newcommand{\PreprocessorTok}[1]{\textcolor[rgb]{0.56,0.35,0.01}{\textit{#1}}}
\newcommand{\RegionMarkerTok}[1]{#1}
\newcommand{\SpecialCharTok}[1]{\textcolor[rgb]{0.81,0.36,0.00}{\textbf{#1}}}
\newcommand{\SpecialStringTok}[1]{\textcolor[rgb]{0.31,0.60,0.02}{#1}}
\newcommand{\StringTok}[1]{\textcolor[rgb]{0.31,0.60,0.02}{#1}}
\newcommand{\VariableTok}[1]{\textcolor[rgb]{0.00,0.00,0.00}{#1}}
\newcommand{\VerbatimStringTok}[1]{\textcolor[rgb]{0.31,0.60,0.02}{#1}}
\newcommand{\WarningTok}[1]{\textcolor[rgb]{0.56,0.35,0.01}{\textbf{\textit{#1}}}}
\usepackage{graphicx}
\makeatletter
\def\maxwidth{\ifdim\Gin@nat@width>\linewidth\linewidth\else\Gin@nat@width\fi}
\def\maxheight{\ifdim\Gin@nat@height>\textheight\textheight\else\Gin@nat@height\fi}
\makeatother
% Scale images if necessary, so that they will not overflow the page
% margins by default, and it is still possible to overwrite the defaults
% using explicit options in \includegraphics[width, height, ...]{}
\setkeys{Gin}{width=\maxwidth,height=\maxheight,keepaspectratio}
% Set default figure placement to htbp
\makeatletter
\def\fps@figure{htbp}
\makeatother
\setlength{\emergencystretch}{3em} % prevent overfull lines
\providecommand{\tightlist}{%
  \setlength{\itemsep}{0pt}\setlength{\parskip}{0pt}}
\setcounter{secnumdepth}{-\maxdimen} % remove section numbering
\usepackage{booktabs}
\usepackage{longtable}
\usepackage{array}
\usepackage{multirow}
\usepackage{wrapfig}
\usepackage{float}
\usepackage{colortbl}
\usepackage{pdflscape}
\usepackage{tabu}
\usepackage{threeparttable}
\usepackage{threeparttablex}
\usepackage[normalem]{ulem}
\usepackage{makecell}
\usepackage{xcolor}
\ifLuaTeX
  \usepackage{selnolig}  % disable illegal ligatures
\fi
\usepackage{bookmark}
\IfFileExists{xurl.sty}{\usepackage{xurl}}{} % add URL line breaks if available
\urlstyle{same}
\hypersetup{
  pdftitle={Geospatial Risk Modeling: Predictive Policing in Chicago},
  pdfauthor={Tutut Indriaty},
  hidelinks,
  pdfcreator={LaTeX via pandoc}}

\title{Geospatial Risk Modeling: Predictive Policing in Chicago}
\author{Tutut Indriaty}
\date{2024-11-01}

\begin{document}
\maketitle

{
\setcounter{tocdepth}{2}
\tableofcontents
}
\section{Introduction}\label{introduction}

\textbf{Predictive policing} is a public sector machine learning
algorithm used to anticipate, prevent, and respond to future crimes more
effectively by analyzing crime-related data. Predicting future crime is
performed by geospatial risk model, a regression model which uses
variables that can help predicting risks of crime happening in the
neighborhood in the future. My outcome of interest for this modeling is
predicting future assaults in the city of Chicago, especially simple
assaults. \textbf{Simple assault} counts when someone knowingly causes
someone else reasonably scared of being hit, such as threats or
perceived threat of harm. Simple assaults may not cause physical harm
directly but still have a significant impact on the sense of safety and
well-being of people. Simple assault also is the most reported assault
in Chicago.

However, choosing simple assaults can result in \textbf{many biases}.
Starts from underreporting, when talking about simple assaults, the
physical proof is none. Many people will choose not to report as they do
not have ``proofs'' or a few of them think (which already makes this
clearly biased) that it is not serious enough to be reported. After
that, different backgrounds, like racial, cultural, and socio-economic
can produce bias data as the wealthier and the more empowered
communities report more than the marginalized ones. Lastly, \textbf{the
choice of predictors}, which we will discuss further, can lead to bias
if it is not apply uniformly across all areas.

\section{Chicago City and Simple
Assaults}\label{chicago-city-and-simple-assaults}

\textbf{a. Setup}

I start with loading all the library packages needed for this modeling
process.

\textbf{b. Outcome of Interest: Simple Assaults as Dependent Variable}

The data of simple assaults of 2017 is taken from open data of the City
of Chicago (\url{data.cityofchicago.org}). The results as shown in the
maps display \textbf{many of reports} of the assault happening across
Chicago with some concentrations in the \textbf{city center, on the west
part of the city, and on the south part.}

\begin{Shaded}
\begin{Highlighting}[]
\NormalTok{assaults }\OtherTok{\textless{}{-}}
  \FunctionTok{read.socrata}\NormalTok{(}\StringTok{"https://data.cityofchicago.org/Public{-}Safety/Crimes{-}2017/d62x{-}nvdr"}\NormalTok{) }\SpecialCharTok{\%\textgreater{}\%} 
    \FunctionTok{filter}\NormalTok{(Primary.Type }\SpecialCharTok{==} \StringTok{"ASSAULT"} \SpecialCharTok{\&}\NormalTok{ Description }\SpecialCharTok{==} \StringTok{"SIMPLE"}\NormalTok{) }\SpecialCharTok{\%\textgreater{}\%}
    \FunctionTok{mutate}\NormalTok{(}\AttributeTok{x =} \FunctionTok{gsub}\NormalTok{(}\StringTok{"[()]"}\NormalTok{, }\StringTok{""}\NormalTok{, Location)) }\SpecialCharTok{\%\textgreater{}\%}
    \FunctionTok{separate}\NormalTok{(x,}\AttributeTok{into=} \FunctionTok{c}\NormalTok{(}\StringTok{"Y"}\NormalTok{,}\StringTok{"X"}\NormalTok{), }\AttributeTok{sep=}\StringTok{","}\NormalTok{) }\SpecialCharTok{\%\textgreater{}\%}
    \FunctionTok{mutate}\NormalTok{(}\AttributeTok{X =} \FunctionTok{as.numeric}\NormalTok{(X),}\AttributeTok{Y =} \FunctionTok{as.numeric}\NormalTok{(Y)) }\SpecialCharTok{\%\textgreater{}\%} 
    \FunctionTok{na.omit}\NormalTok{() }\SpecialCharTok{\%\textgreater{}\%}
    \FunctionTok{st\_as\_sf}\NormalTok{(}\AttributeTok{coords =} \FunctionTok{c}\NormalTok{(}\StringTok{"X"}\NormalTok{, }\StringTok{"Y"}\NormalTok{), }\AttributeTok{crs =} \DecValTok{4326}\NormalTok{, }\AttributeTok{agr =} \StringTok{"constant"}\NormalTok{)}\SpecialCharTok{\%\textgreater{}\%}
    \FunctionTok{st\_transform}\NormalTok{(}\StringTok{\textquotesingle{}ESRI:102271\textquotesingle{}}\NormalTok{) }\SpecialCharTok{\%\textgreater{}\%} 
    \FunctionTok{distinct}\NormalTok{()}

\NormalTok{chicagoBoundary }\OtherTok{\textless{}{-}} 
  \FunctionTok{st\_read}\NormalTok{(}\FunctionTok{file.path}\NormalTok{(root.dir,}\StringTok{"/Chapter5/chicagoBoundary.geojson"}\NormalTok{)) }\SpecialCharTok{\%\textgreater{}\%}
  \FunctionTok{st\_transform}\NormalTok{(}\StringTok{\textquotesingle{}ESRI:102271\textquotesingle{}}\NormalTok{)}

\NormalTok{assaults\_clipped }\OtherTok{\textless{}{-}} \FunctionTok{st\_intersection}\NormalTok{(assaults, chicagoBoundary)}
\end{Highlighting}
\end{Shaded}

\begin{Shaded}
\begin{Highlighting}[]
\FunctionTok{grid.arrange}\NormalTok{(}\AttributeTok{ncol =} \DecValTok{2}\NormalTok{,}
  \FunctionTok{ggplot}\NormalTok{() }\SpecialCharTok{+} 
    \FunctionTok{geom\_sf}\NormalTok{(}\AttributeTok{data =}\NormalTok{ chicagoBoundary, }\AttributeTok{fill =} \StringTok{"lightblue"}\NormalTok{, }\AttributeTok{color =} \StringTok{"black"}\NormalTok{) }\SpecialCharTok{+} 
    \FunctionTok{geom\_sf}\NormalTok{(}\AttributeTok{data =}\NormalTok{ assaults\_clipped, }\AttributeTok{colour =} \StringTok{"red"}\NormalTok{, }\AttributeTok{size =} \FloatTok{0.3}\NormalTok{, }\AttributeTok{show.legend =} \StringTok{"point"}\NormalTok{, }\AttributeTok{alpha =} \FloatTok{0.6}\NormalTok{) }\SpecialCharTok{+} 
    \FunctionTok{labs}\NormalTok{(}\AttributeTok{title =} \StringTok{"Assaults in Chicago {-} 2017"}\NormalTok{) }\SpecialCharTok{+}
    \FunctionTok{mapTheme}\NormalTok{(}\AttributeTok{title\_size =} \DecValTok{12}\NormalTok{) }\SpecialCharTok{+}
    \FunctionTok{theme}\NormalTok{(}
      \AttributeTok{panel.background =} \FunctionTok{element\_rect}\NormalTok{(}\AttributeTok{fill =} \StringTok{"white"}\NormalTok{),}
      \AttributeTok{plot.title =} \FunctionTok{element\_text}\NormalTok{(}\AttributeTok{hjust =} \FloatTok{0.5}\NormalTok{, }\AttributeTok{face =} \StringTok{"bold"}\NormalTok{),}
      \AttributeTok{axis.text =} \FunctionTok{element\_blank}\NormalTok{(),}
      \AttributeTok{axis.ticks =} \FunctionTok{element\_blank}\NormalTok{()),}
  \FunctionTok{ggplot}\NormalTok{() }\SpecialCharTok{+} 
    \FunctionTok{geom\_sf}\NormalTok{(}\AttributeTok{data =}\NormalTok{ chicagoBoundary, }\AttributeTok{fill =} \StringTok{"grey20"}\NormalTok{, }\AttributeTok{color =} \StringTok{"white"}\NormalTok{) }\SpecialCharTok{+} 
    \FunctionTok{stat\_density2d}\NormalTok{(}
      \AttributeTok{data =} \FunctionTok{data.frame}\NormalTok{(}\FunctionTok{st\_coordinates}\NormalTok{(assaults)), }
      \FunctionTok{aes}\NormalTok{(X, Y, }\AttributeTok{fill =}\NormalTok{ ..level.., }\AttributeTok{alpha =}\NormalTok{ ..level..), }
      \AttributeTok{size =} \FloatTok{0.2}\NormalTok{, }\AttributeTok{bins =} \DecValTok{40}\NormalTok{, }\AttributeTok{geom =} \StringTok{\textquotesingle{}polygon\textquotesingle{}}\NormalTok{) }\SpecialCharTok{+} 
    \FunctionTok{scale\_fill\_viridis}\NormalTok{(}\AttributeTok{option =} \StringTok{"plasma"}\NormalTok{, }\AttributeTok{direction =} \SpecialCharTok{{-}}\DecValTok{1}\NormalTok{) }\SpecialCharTok{+} 
    \FunctionTok{scale\_alpha}\NormalTok{(}\AttributeTok{range =} \FunctionTok{c}\NormalTok{(}\FloatTok{0.00}\NormalTok{, }\FloatTok{0.5}\NormalTok{), }\AttributeTok{guide =} \ConstantTok{FALSE}\NormalTok{) }\SpecialCharTok{+} 
    \FunctionTok{labs}\NormalTok{(}\AttributeTok{title =} \StringTok{"Density of Assaults"}\NormalTok{) }\SpecialCharTok{+}
    \FunctionTok{mapTheme}\NormalTok{(}\AttributeTok{title\_size =} \DecValTok{12}\NormalTok{) }\SpecialCharTok{+}
    \FunctionTok{theme}\NormalTok{(}
      \AttributeTok{legend.position =} \FunctionTok{c}\NormalTok{(}\FloatTok{0.15}\NormalTok{, }\FloatTok{0.125}\NormalTok{),}
      \AttributeTok{legend.title =} \FunctionTok{element\_text}\NormalTok{(}\AttributeTok{size =} \DecValTok{4}\NormalTok{),}
      \AttributeTok{legend.text =} \FunctionTok{element\_text}\NormalTok{(}\AttributeTok{size =} \DecValTok{4}\NormalTok{),}
      \AttributeTok{legend.key.width =} \FunctionTok{unit}\NormalTok{(}\DecValTok{1}\NormalTok{, }\StringTok{"cm"}\NormalTok{),}
      \AttributeTok{legend.key.height =} \FunctionTok{unit}\NormalTok{(}\FloatTok{0.25}\NormalTok{, }\StringTok{"cm"}\NormalTok{),}
      \AttributeTok{legend.background =} \FunctionTok{element\_rect}\NormalTok{(}\AttributeTok{fill =} \FunctionTok{alpha}\NormalTok{(}\StringTok{\textquotesingle{}white\textquotesingle{}}\NormalTok{, }\FloatTok{0.7}\NormalTok{), }\AttributeTok{color =} \ConstantTok{NA}\NormalTok{),}
      \AttributeTok{panel.background =} \FunctionTok{element\_rect}\NormalTok{(}\AttributeTok{fill =} \StringTok{"white"}\NormalTok{),}
      \AttributeTok{plot.title =} \FunctionTok{element\_text}\NormalTok{(}\AttributeTok{hjust =} \FloatTok{0.5}\NormalTok{, }\AttributeTok{face =} \StringTok{"bold"}\NormalTok{),}
      \AttributeTok{axis.text =} \FunctionTok{element\_blank}\NormalTok{(),}
      \AttributeTok{axis.ticks =} \FunctionTok{element\_blank}\NormalTok{(),}
      \AttributeTok{panel.grid.major =} \FunctionTok{element\_line}\NormalTok{(),}
      \AttributeTok{panel.grid.minor =} \FunctionTok{element\_blank}\NormalTok{()))}
\end{Highlighting}
\end{Shaded}

\includegraphics{Indriaty_TututHW3_files/figure-latex/chicagoboundary-1.pdf}

Using the fishnet grid, we could see the number of assaults per grid
cell. The map below shows that \textbf{most assaults happen in the city
center} with one grid cell of 500 m x 500 m showing assault numbers
above 60 , followed by the south part with the most is in the range of
40 to 60 assaults and the west part of the city with the most is in the
range of 20 to 40 assaults. From this we also recognize that the
assaults are clustering.

\begin{Shaded}
\begin{Highlighting}[]
\NormalTok{fishnet }\OtherTok{\textless{}{-}} 
  \FunctionTok{st\_make\_grid}\NormalTok{(chicagoBoundary,}
               \AttributeTok{cellsize =} \DecValTok{500}\NormalTok{, }
               \AttributeTok{square =} \ConstantTok{TRUE}\NormalTok{) }\SpecialCharTok{\%\textgreater{}\%}
\NormalTok{  .[chicagoBoundary] }\SpecialCharTok{\%\textgreater{}\%}
  \FunctionTok{st\_sf}\NormalTok{() }\SpecialCharTok{\%\textgreater{}\%}
  \FunctionTok{mutate}\NormalTok{(}\AttributeTok{uniqueID =} \DecValTok{1}\SpecialCharTok{:}\FunctionTok{n}\NormalTok{())}
\NormalTok{crime\_net }\OtherTok{\textless{}{-}} 
\NormalTok{  dplyr}\SpecialCharTok{::}\FunctionTok{select}\NormalTok{(assaults\_clipped) }\SpecialCharTok{\%\textgreater{}\%} 
  \FunctionTok{mutate}\NormalTok{(}\AttributeTok{countAssaults =} \DecValTok{1}\NormalTok{) }\SpecialCharTok{\%\textgreater{}\%} 
  \FunctionTok{aggregate}\NormalTok{(., fishnet, sum) }\SpecialCharTok{\%\textgreater{}\%}
  \FunctionTok{mutate}\NormalTok{(}\AttributeTok{countAssaults =} \FunctionTok{replace\_na}\NormalTok{(countAssaults, }\DecValTok{0}\NormalTok{),}
         \AttributeTok{uniqueID =} \DecValTok{1}\SpecialCharTok{:}\FunctionTok{n}\NormalTok{(),}
         \AttributeTok{cvID =} \FunctionTok{sample}\NormalTok{(}\FunctionTok{round}\NormalTok{(}\FunctionTok{nrow}\NormalTok{(fishnet) }\SpecialCharTok{/} \DecValTok{24}\NormalTok{), }
                       \AttributeTok{size=}\FunctionTok{nrow}\NormalTok{(fishnet), }\AttributeTok{replace =} \ConstantTok{TRUE}\NormalTok{))}


\FunctionTok{grid.arrange}\NormalTok{(}\AttributeTok{ncol =} \DecValTok{2}\NormalTok{,}
\FunctionTok{ggplot}\NormalTok{() }\SpecialCharTok{+}
  \FunctionTok{geom\_sf}\NormalTok{(}\AttributeTok{data =}\NormalTok{ chicagoBoundary, }\AttributeTok{fill =} \StringTok{"white"}\NormalTok{, }\AttributeTok{color =} \StringTok{"grey"}\NormalTok{) }\SpecialCharTok{+}
  \FunctionTok{geom\_sf}\NormalTok{(}\AttributeTok{data =}\NormalTok{ fishnet, }\AttributeTok{color =} \StringTok{"black"}\NormalTok{, }\AttributeTok{fill =} \ConstantTok{NA}\NormalTok{, }\AttributeTok{alpha =} \FloatTok{0.7}\NormalTok{) }\SpecialCharTok{+}
  \FunctionTok{labs}\NormalTok{(}\AttributeTok{title =} \StringTok{"Fishnet Grid over Chicago"}\NormalTok{,}
       \AttributeTok{subtitle =} \StringTok{"Grid cell size: 500m x 500m"}\NormalTok{) }\SpecialCharTok{+}
  \FunctionTok{theme\_minimal}\NormalTok{() }\SpecialCharTok{+}
  \FunctionTok{theme}\NormalTok{(}
    \AttributeTok{plot.title =} \FunctionTok{element\_text}\NormalTok{(}\AttributeTok{hjust =} \FloatTok{0.5}\NormalTok{, }\AttributeTok{size =} \DecValTok{12}\NormalTok{, }\AttributeTok{face =} \StringTok{"bold"}\NormalTok{),}
    \AttributeTok{plot.subtitle =} \FunctionTok{element\_text}\NormalTok{(}\AttributeTok{hjust =} \FloatTok{0.5}\NormalTok{, }\AttributeTok{size =} \DecValTok{8}\NormalTok{),}
    \AttributeTok{axis.text =} \FunctionTok{element\_blank}\NormalTok{(),}
    \AttributeTok{axis.ticks =} \FunctionTok{element\_blank}\NormalTok{(),}
    \AttributeTok{panel.grid =} \FunctionTok{element\_blank}\NormalTok{()),}
\FunctionTok{ggplot}\NormalTok{() }\SpecialCharTok{+}
  \FunctionTok{geom\_sf}\NormalTok{(}\AttributeTok{data =}\NormalTok{ crime\_net, }\FunctionTok{aes}\NormalTok{(}\AttributeTok{fill =}\NormalTok{ countAssaults), }\AttributeTok{color =} \ConstantTok{NA}\NormalTok{) }\SpecialCharTok{+}
  \FunctionTok{geom\_sf}\NormalTok{(}\AttributeTok{data =}\NormalTok{ fishnet, }\AttributeTok{color =} \StringTok{"black"}\NormalTok{, }\AttributeTok{fill =} \ConstantTok{NA}\NormalTok{, }\AttributeTok{alpha =} \FloatTok{0.7}\NormalTok{) }\SpecialCharTok{+}
  \FunctionTok{scale\_fill\_viridis}\NormalTok{(}\AttributeTok{option =} \StringTok{"plasma"}\NormalTok{, }\AttributeTok{direction =} \SpecialCharTok{{-}}\DecValTok{1}\NormalTok{, }\AttributeTok{name =} \StringTok{"Assaults Count"}\NormalTok{, }
                     \AttributeTok{guide =} \FunctionTok{guide\_colorbar}\NormalTok{(}\AttributeTok{barwidth =} \FloatTok{0.5}\NormalTok{, }\AttributeTok{barheight =} \DecValTok{4}\NormalTok{)) }\SpecialCharTok{+} 
  \FunctionTok{labs}\NormalTok{(}\AttributeTok{title =} \StringTok{"Assaults per Grid Cell"}\NormalTok{, }\AttributeTok{subtitle =} \StringTok{"Based on fishnet grid"}\NormalTok{) }\SpecialCharTok{+}
  \FunctionTok{theme\_minimal}\NormalTok{() }\SpecialCharTok{+}
  \FunctionTok{theme}\NormalTok{(}
    \AttributeTok{plot.title =} \FunctionTok{element\_text}\NormalTok{(}\AttributeTok{hjust =} \FloatTok{0.5}\NormalTok{, }\AttributeTok{size =} \DecValTok{12}\NormalTok{, }\AttributeTok{face =} \StringTok{"bold"}\NormalTok{),}
    \AttributeTok{plot.subtitle =} \FunctionTok{element\_text}\NormalTok{(}\AttributeTok{hjust =} \FloatTok{0.5}\NormalTok{, }\AttributeTok{size =} \DecValTok{8}\NormalTok{),}
    \AttributeTok{legend.direction =} \StringTok{"vertical"}\NormalTok{,}
    \AttributeTok{legend.position =} \FunctionTok{c}\NormalTok{(}\FloatTok{0.15}\NormalTok{, }\FloatTok{0.125}\NormalTok{),}
    \AttributeTok{legend.title =} \FunctionTok{element\_text}\NormalTok{(}\AttributeTok{size =} \DecValTok{4}\NormalTok{),}
    \AttributeTok{legend.text =} \FunctionTok{element\_text}\NormalTok{(}\AttributeTok{size =} \DecValTok{4}\NormalTok{),}
    \AttributeTok{legend.key.width =} \FunctionTok{unit}\NormalTok{(}\FloatTok{0.25}\NormalTok{, }\StringTok{"cm"}\NormalTok{),}
    \AttributeTok{legend.key.height =} \FunctionTok{unit}\NormalTok{(}\DecValTok{2}\NormalTok{, }\StringTok{"cm"}\NormalTok{),}
    \AttributeTok{axis.text =} \FunctionTok{element\_blank}\NormalTok{(),}
    \AttributeTok{axis.ticks =} \FunctionTok{element\_blank}\NormalTok{(),}
    \AttributeTok{panel.grid =} \FunctionTok{element\_blank}\NormalTok{()))}
\end{Highlighting}
\end{Shaded}

\includegraphics{Indriaty_TututHW3_files/figure-latex/fishnet-1.pdf}

\section{Risk Factors as Independent
Variables}\label{risk-factors-as-independent-variables}

\textbf{a. Chosen Risk Factors}

The \textbf{risk factors} chosen for simple assaults are:

\begin{itemize}
\item
  \textbf{Metra stations} which is Chicago's rail network as public
  places where legal authorities are not around every time and are
  mostly underground,
\item
  \textbf{CCTV or POD (Police Observation Device) camera} as one of the
  supposedly proofs of what happened in the location of crime,
\item
  \textbf{Street Lights Out} which is the complaints of people in the
  neighborhoods of street lights not functioned properly or broken as an
  opportunity of crime taking place in the dark, and
\item
  \textbf{Liquor Retail} as places that can possibly sell liquor almost
  24 hours and open in all neighborhoods not just the big ones with many
  people may possibly be gathering, high-stress, drunk, and
  half-conscious.
\end{itemize}

We clean and wrangle the data like choosing useful fields, transforming
into the same projections as the dependent variable and combining the
factors with the assaults data. The small multiple maps below show the
two risk factors of Liquor Retails and Street Lights Out spread across
Chicago city and the other two of Metro Stops and POD Cameras clustered
in some areas that can lead to prediction bias.

\begin{Shaded}
\begin{Highlighting}[]
\NormalTok{metraStops }\OtherTok{\textless{}{-}} \FunctionTok{st\_read}\NormalTok{(}\StringTok{"data/MetraStations.shp"}\NormalTok{) }\SpecialCharTok{\%\textgreater{}\%}
                \FunctionTok{filter}\NormalTok{(MUNICIPALI }\SpecialCharTok{==} \StringTok{"Chicago"}\NormalTok{) }\SpecialCharTok{\%\textgreater{}\%}
                \FunctionTok{st\_as\_sf}\NormalTok{(}\AttributeTok{coords =} \FunctionTok{c}\NormalTok{(}\StringTok{"X"}\NormalTok{, }\StringTok{"Y"}\NormalTok{), }\AttributeTok{crs =} \DecValTok{4326}\NormalTok{, }\AttributeTok{agr =} \StringTok{"constant"}\NormalTok{) }\SpecialCharTok{\%\textgreater{}\%}
                \FunctionTok{st\_transform}\NormalTok{(}\FunctionTok{st\_crs}\NormalTok{(fishnet)) }\SpecialCharTok{\%\textgreater{}\%}
\NormalTok{                dplyr}\SpecialCharTok{::}\FunctionTok{select}\NormalTok{(geometry) }\SpecialCharTok{\%\textgreater{}\%}
                \FunctionTok{mutate}\NormalTok{(}\AttributeTok{Legend =} \StringTok{"Metra Stops"}\NormalTok{)}
\NormalTok{metraStops }\OtherTok{\textless{}{-}}  \FunctionTok{st\_cast}\NormalTok{(metraStops, }\StringTok{"POINT"}\NormalTok{)}

\NormalTok{cctv }\OtherTok{\textless{}{-}} \FunctionTok{st\_read}\NormalTok{(}\StringTok{"data/chicago\_pod\_cameras\_031816.geojson"}\NormalTok{) }\SpecialCharTok{\%\textgreater{}\%}
          \FunctionTok{st\_as\_sf}\NormalTok{(}\AttributeTok{coords =} \FunctionTok{c}\NormalTok{(}\StringTok{"X"}\NormalTok{, }\StringTok{"Y"}\NormalTok{), }\AttributeTok{crs =} \DecValTok{4326}\NormalTok{, }\AttributeTok{agr =} \StringTok{"constant"}\NormalTok{) }\SpecialCharTok{\%\textgreater{}\%}
          \FunctionTok{st\_transform}\NormalTok{(}\FunctionTok{st\_crs}\NormalTok{(fishnet)) }\SpecialCharTok{\%\textgreater{}\%}
\NormalTok{          dplyr}\SpecialCharTok{::}\FunctionTok{select}\NormalTok{(geometry) }\SpecialCharTok{\%\textgreater{}\%}
          \FunctionTok{mutate}\NormalTok{(}\AttributeTok{Legend =} \StringTok{"POD Cameras"}\NormalTok{)}

\NormalTok{streetLightsOut }\OtherTok{\textless{}{-}} 
  \FunctionTok{read.socrata}\NormalTok{(}\StringTok{"https://data.cityofchicago.org/Service{-}Requests/311{-}Service{-}Requests{-}Street{-}Lights{-}All{-}Out/zuxi{-}7xem"}\NormalTok{) }\SpecialCharTok{\%\textgreater{}\%}
    \FunctionTok{mutate}\NormalTok{(}\AttributeTok{year =} \FunctionTok{substr}\NormalTok{(creation\_date,}\DecValTok{1}\NormalTok{,}\DecValTok{4}\NormalTok{)) }\SpecialCharTok{\%\textgreater{}\%} \FunctionTok{filter}\NormalTok{(year }\SpecialCharTok{==} \StringTok{"2017"}\NormalTok{) }\SpecialCharTok{\%\textgreater{}\%}
\NormalTok{    dplyr}\SpecialCharTok{::}\FunctionTok{select}\NormalTok{(}\AttributeTok{Y =}\NormalTok{ latitude, }\AttributeTok{X =}\NormalTok{ longitude) }\SpecialCharTok{\%\textgreater{}\%}
    \FunctionTok{na.omit}\NormalTok{() }\SpecialCharTok{\%\textgreater{}\%}
    \FunctionTok{st\_as\_sf}\NormalTok{(}\AttributeTok{coords =} \FunctionTok{c}\NormalTok{(}\StringTok{"X"}\NormalTok{, }\StringTok{"Y"}\NormalTok{), }\AttributeTok{crs =} \DecValTok{4326}\NormalTok{, }\AttributeTok{agr =} \StringTok{"constant"}\NormalTok{) }\SpecialCharTok{\%\textgreater{}\%}
    \FunctionTok{st\_transform}\NormalTok{(}\FunctionTok{st\_crs}\NormalTok{(fishnet)) }\SpecialCharTok{\%\textgreater{}\%}
    \FunctionTok{mutate}\NormalTok{(}\AttributeTok{Legend =} \StringTok{"Street\_Lights\_Out"}\NormalTok{)}

\NormalTok{liquorRetail }\OtherTok{\textless{}{-}} 
  \FunctionTok{read.socrata}\NormalTok{(}\StringTok{"https://data.cityofchicago.org/resource/nrmj{-}3kcf.json"}\NormalTok{) }\SpecialCharTok{\%\textgreater{}\%}  
    \FunctionTok{filter}\NormalTok{(business\_activity }\SpecialCharTok{==} \StringTok{"Retail Sales of Packaged Liquor"}\NormalTok{) }\SpecialCharTok{\%\textgreater{}\%}
\NormalTok{    dplyr}\SpecialCharTok{::}\FunctionTok{select}\NormalTok{(}\AttributeTok{Y =}\NormalTok{ latitude, }\AttributeTok{X =}\NormalTok{ longitude) }\SpecialCharTok{\%\textgreater{}\%}
    \FunctionTok{na.omit}\NormalTok{() }\SpecialCharTok{\%\textgreater{}\%}
    \FunctionTok{st\_as\_sf}\NormalTok{(}\AttributeTok{coords =} \FunctionTok{c}\NormalTok{(}\StringTok{"X"}\NormalTok{, }\StringTok{"Y"}\NormalTok{), }\AttributeTok{crs =} \DecValTok{4326}\NormalTok{, }\AttributeTok{agr =} \StringTok{"constant"}\NormalTok{) }\SpecialCharTok{\%\textgreater{}\%}
    \FunctionTok{st\_transform}\NormalTok{(}\FunctionTok{st\_crs}\NormalTok{(fishnet)) }\SpecialCharTok{\%\textgreater{}\%}
    \FunctionTok{mutate}\NormalTok{(}\AttributeTok{Legend =} \StringTok{"Liquor\_Retail"}\NormalTok{)}

\NormalTok{neighborhoods }\OtherTok{\textless{}{-}} 
  \FunctionTok{st\_read}\NormalTok{(}\StringTok{"https://raw.githubusercontent.com/blackmad/neighborhoods/master/chicago.geojson"}\NormalTok{) }\SpecialCharTok{\%\textgreater{}\%}
  \FunctionTok{st\_transform}\NormalTok{(}\FunctionTok{st\_crs}\NormalTok{(fishnet)) }

\NormalTok{combined\_sf }\OtherTok{\textless{}{-}} \FunctionTok{rbind}\NormalTok{(metraStops, cctv, streetLightsOut, liquorRetail)}

\NormalTok{chicago\_base\_map }\OtherTok{\textless{}{-}} \FunctionTok{ggplot}\NormalTok{() }\SpecialCharTok{+}
  \FunctionTok{geom\_sf}\NormalTok{(}\AttributeTok{data =}\NormalTok{ chicagoBoundary, }\AttributeTok{fill =} \StringTok{"white"}\NormalTok{, }\AttributeTok{color =} \StringTok{"black"}\NormalTok{) }\SpecialCharTok{+}
  \FunctionTok{theme\_void}\NormalTok{()}

\NormalTok{combined\_map }\OtherTok{\textless{}{-}}\NormalTok{ chicago\_base\_map }\SpecialCharTok{+}
  \FunctionTok{geom\_sf}\NormalTok{(}\AttributeTok{data =}\NormalTok{ combined\_sf, }\FunctionTok{aes}\NormalTok{(}\AttributeTok{color =}\NormalTok{ Legend), }\AttributeTok{size =} \FloatTok{0.25}\NormalTok{, }\AttributeTok{alpha =} \FloatTok{0.7}\NormalTok{) }\SpecialCharTok{+}
  \FunctionTok{facet\_wrap}\NormalTok{(}\SpecialCharTok{\textasciitilde{}}\NormalTok{ Legend) }\SpecialCharTok{+}
  \FunctionTok{labs}\NormalTok{(}\AttributeTok{title =} \StringTok{"Risk Factors, Points"}\NormalTok{) }\SpecialCharTok{+}
  \FunctionTok{theme}\NormalTok{(}\AttributeTok{legend.position =} \StringTok{"none"}\NormalTok{, }
        \AttributeTok{strip.background =} \FunctionTok{element\_rect}\NormalTok{(}\AttributeTok{fill =} \StringTok{"gray"}\NormalTok{),}
        \AttributeTok{strip.text =} \FunctionTok{element\_text}\NormalTok{(}\AttributeTok{size =} \DecValTok{10}\NormalTok{, }\AttributeTok{face =} \StringTok{"bold"}\NormalTok{))}

\FunctionTok{print}\NormalTok{(combined\_map)}
\end{Highlighting}
\end{Shaded}

\includegraphics{Indriaty_TututHW3_files/figure-latex/riskfactors-1.pdf}

For these maps, the risk factors are combined with fishnet grid data
therefore displaying the variables in grid cell with lighter colors as
the areas with the most appeared factors. \textbf{Metra Stops map will
probably show bias as the covered areas are not many} therefore can lead
to a believe that the other areas without them are safer/less risk. It
also applies to Liquor Retail and POD Cameras.

\begin{Shaded}
\begin{Highlighting}[]
\NormalTok{vars\_net }\OtherTok{\textless{}{-}}\NormalTok{ combined\_sf }\SpecialCharTok{\%\textgreater{}\%}
  \FunctionTok{st\_join}\NormalTok{(fishnet, }\AttributeTok{join=}\NormalTok{st\_within) }\SpecialCharTok{\%\textgreater{}\%}
  \FunctionTok{st\_drop\_geometry}\NormalTok{() }\SpecialCharTok{\%\textgreater{}\%}
  \FunctionTok{group\_by}\NormalTok{(uniqueID, Legend) }\SpecialCharTok{\%\textgreater{}\%}
  \FunctionTok{summarize}\NormalTok{(}\AttributeTok{count =} \FunctionTok{n}\NormalTok{()) }\SpecialCharTok{\%\textgreater{}\%}
  \FunctionTok{left\_join}\NormalTok{(fishnet, ., }\AttributeTok{by =} \StringTok{"uniqueID"}\NormalTok{) }\SpecialCharTok{\%\textgreater{}\%}
  \FunctionTok{spread}\NormalTok{(Legend, count, }\AttributeTok{fill=}\DecValTok{0}\NormalTok{) }\SpecialCharTok{\%\textgreater{}\%}
\NormalTok{  dplyr}\SpecialCharTok{::}\FunctionTok{select}\NormalTok{(}\SpecialCharTok{{-}}\StringTok{\textasciigrave{}}\AttributeTok{\textless{}NA\textgreater{}}\StringTok{\textasciigrave{}}\NormalTok{) }\SpecialCharTok{\%\textgreater{}\%}
  \FunctionTok{ungroup}\NormalTok{()}

\NormalTok{vars\_net.long }\OtherTok{\textless{}{-}} 
  \FunctionTok{gather}\NormalTok{(vars\_net, Variable, value, }\SpecialCharTok{{-}}\NormalTok{geometry, }\SpecialCharTok{{-}}\NormalTok{uniqueID)}

\NormalTok{vars }\OtherTok{\textless{}{-}} \FunctionTok{unique}\NormalTok{(vars\_net.long}\SpecialCharTok{$}\NormalTok{Variable)}
\NormalTok{mapList }\OtherTok{\textless{}{-}} \FunctionTok{list}\NormalTok{()}

\ControlFlowTok{for}\NormalTok{(i }\ControlFlowTok{in}\NormalTok{ vars)\{}
\NormalTok{  mapList[[i]] }\OtherTok{\textless{}{-}} 
    \FunctionTok{ggplot}\NormalTok{() }\SpecialCharTok{+}
      \FunctionTok{geom\_sf}\NormalTok{(}\AttributeTok{data =} \FunctionTok{filter}\NormalTok{(vars\_net.long, Variable }\SpecialCharTok{==}\NormalTok{ i), }\FunctionTok{aes}\NormalTok{(}\AttributeTok{fill=}\NormalTok{value), }\AttributeTok{colour=}\ConstantTok{NA}\NormalTok{) }\SpecialCharTok{+}
      \FunctionTok{scale\_fill\_viridis}\NormalTok{(}\AttributeTok{option =} \StringTok{"plasma"}\NormalTok{, }\AttributeTok{name=}\StringTok{""}\NormalTok{) }\SpecialCharTok{+}
      \FunctionTok{labs}\NormalTok{(}\AttributeTok{title=}\NormalTok{i) }\SpecialCharTok{+}
      \FunctionTok{mapTheme}\NormalTok{()\}}

\FunctionTok{do.call}\NormalTok{(grid.arrange,}\FunctionTok{c}\NormalTok{(mapList, }\AttributeTok{ncol=}\DecValTok{2}\NormalTok{, }\AttributeTok{top=}\StringTok{"Risk Factors by Fishnet"}\NormalTok{))}
\end{Highlighting}
\end{Shaded}

\includegraphics{Indriaty_TututHW3_files/figure-latex/riskfactors2-1.pdf}

\textbf{b. Correlative Analysis}

To know the correlation between dependent and independent variables, we
use \textbf{scatterplots} to see the relation. According to the
scatterplots below, all independent variables have positive relations
with the dependent variable. \textbf{It is interesting especially for
the risk factor of number of POD Cameras}. As the number of POD cameras
increase, the number of the assault also increase. This could happen
because of the areas crime history which resulted in the camera
installment for the high-risk areas.

\begin{Shaded}
\begin{Highlighting}[]
\NormalTok{combined }\OtherTok{\textless{}{-}} \FunctionTok{cbind}\NormalTok{(crime\_net, vars\_net) }\SpecialCharTok{\%\textgreater{}\%}
\NormalTok{            dplyr}\SpecialCharTok{::}\FunctionTok{select}\NormalTok{(countAssaults, uniqueID, cvID, Liquor\_Retail,}
\NormalTok{                          Metra.Stops, POD.Cameras, Street\_Lights\_Out) }\SpecialCharTok{\%\textgreater{}\%}
            \FunctionTok{st\_drop\_geometry}\NormalTok{()}

\CommentTok{\#MetraStops}
\NormalTok{plot1 }\OtherTok{\textless{}{-}} \FunctionTok{ggplot}\NormalTok{(combined, }\FunctionTok{aes}\NormalTok{(}\AttributeTok{x =} \StringTok{\textasciigrave{}}\AttributeTok{Metra.Stops}\StringTok{\textasciigrave{}}\NormalTok{, }\AttributeTok{y =} \StringTok{\textasciigrave{}}\AttributeTok{countAssaults}\StringTok{\textasciigrave{}}\NormalTok{)) }\SpecialCharTok{+} 
  \FunctionTok{geom\_point}\NormalTok{(}\AttributeTok{color =} \StringTok{"dodgerblue"}\NormalTok{, }\AttributeTok{alpha =} \FloatTok{0.6}\NormalTok{, }\AttributeTok{size =} \DecValTok{3}\NormalTok{) }\SpecialCharTok{+}
  \FunctionTok{geom\_smooth}\NormalTok{(}\AttributeTok{method =} \StringTok{"lm"}\NormalTok{, }\AttributeTok{color =} \StringTok{"red"}\NormalTok{, }\AttributeTok{se =} \ConstantTok{FALSE}\NormalTok{, }\AttributeTok{linetype =} \StringTok{"dashed"}\NormalTok{) }\SpecialCharTok{+}
  \FunctionTok{labs}\NormalTok{(}
    \AttributeTok{title =} \StringTok{"Metra Stops vs Assaults Count"}\NormalTok{,  }
    \AttributeTok{x =} \StringTok{"Metra Stops"}\NormalTok{,}
    \AttributeTok{y =} \StringTok{"Number of Assaults"}\NormalTok{,}
    \AttributeTok{caption =} \StringTok{"Figure 3. Relationship between Metra Stops and Assaults Count"}
\NormalTok{  ) }\SpecialCharTok{+} 
  \FunctionTok{scale\_y\_continuous}\NormalTok{() }\SpecialCharTok{+} 
  \FunctionTok{theme\_minimal}\NormalTok{(}\AttributeTok{base\_size =} \DecValTok{12}\NormalTok{) }\SpecialCharTok{+}
  \FunctionTok{theme}\NormalTok{(}
    \AttributeTok{plot.title =} \FunctionTok{element\_text}\NormalTok{(}\AttributeTok{face =} \StringTok{"bold"}\NormalTok{, }\AttributeTok{hjust =} \FloatTok{0.5}\NormalTok{, }\AttributeTok{size =} \DecValTok{10}\NormalTok{, }\AttributeTok{color =} \StringTok{"darkblue"}\NormalTok{),}
    \AttributeTok{plot.caption =} \FunctionTok{element\_text}\NormalTok{(}\AttributeTok{hjust =} \DecValTok{1}\NormalTok{, }\AttributeTok{face =} \StringTok{"italic"}\NormalTok{, }\AttributeTok{size =} \DecValTok{4}\NormalTok{, }\AttributeTok{color =} \StringTok{"darkgray"}\NormalTok{),}
    \AttributeTok{axis.title.x =} \FunctionTok{element\_text}\NormalTok{(}\AttributeTok{face =} \StringTok{"bold"}\NormalTok{, }\AttributeTok{color =} \StringTok{"darkblue"}\NormalTok{, }\AttributeTok{size =} \DecValTok{6}\NormalTok{),}
    \AttributeTok{axis.title.y =} \FunctionTok{element\_text}\NormalTok{(}\AttributeTok{face =} \StringTok{"bold"}\NormalTok{, }\AttributeTok{color =} \StringTok{"darkblue"}\NormalTok{, }\AttributeTok{size =} \DecValTok{6}\NormalTok{),}
    \AttributeTok{axis.text =} \FunctionTok{element\_text}\NormalTok{(}\AttributeTok{size =} \DecValTok{6}\NormalTok{, }\AttributeTok{color =} \StringTok{"black"}\NormalTok{),}
    \AttributeTok{panel.grid.major =} \FunctionTok{element\_line}\NormalTok{(}\AttributeTok{color =} \StringTok{"lightgray"}\NormalTok{, }\AttributeTok{linetype =} \StringTok{"dashed"}\NormalTok{),}
    \AttributeTok{panel.grid.minor =} \FunctionTok{element\_blank}\NormalTok{(),}
    \AttributeTok{plot.background =} \FunctionTok{element\_rect}\NormalTok{(}\AttributeTok{fill =} \StringTok{"white"}\NormalTok{, }\AttributeTok{color =} \ConstantTok{NA}\NormalTok{)}
\NormalTok{  )}

\CommentTok{\#Liquor Retail}
\NormalTok{plot2 }\OtherTok{\textless{}{-}} \FunctionTok{ggplot}\NormalTok{(combined, }\FunctionTok{aes}\NormalTok{(}\AttributeTok{x =} \StringTok{\textasciigrave{}}\AttributeTok{Liquor\_Retail}\StringTok{\textasciigrave{}}\NormalTok{, }\AttributeTok{y =} \StringTok{\textasciigrave{}}\AttributeTok{countAssaults}\StringTok{\textasciigrave{}}\NormalTok{)) }\SpecialCharTok{+} 
  \FunctionTok{geom\_point}\NormalTok{(}\AttributeTok{color =} \StringTok{"dodgerblue"}\NormalTok{, }\AttributeTok{alpha =} \FloatTok{0.6}\NormalTok{, }\AttributeTok{size =} \DecValTok{3}\NormalTok{) }\SpecialCharTok{+}
  \FunctionTok{geom\_smooth}\NormalTok{(}\AttributeTok{method =} \StringTok{"lm"}\NormalTok{, }\AttributeTok{color =} \StringTok{"red"}\NormalTok{, }\AttributeTok{se =} \ConstantTok{FALSE}\NormalTok{, }\AttributeTok{linetype =} \StringTok{"dashed"}\NormalTok{) }\SpecialCharTok{+}
  \FunctionTok{labs}\NormalTok{(}
    \AttributeTok{title =} \StringTok{"Liquor Retail vs Assaults Count"}\NormalTok{,  }
    \AttributeTok{x =} \StringTok{"Number of Liquor Retail"}\NormalTok{,}
    \AttributeTok{y =} \StringTok{"Number of Assaults"}\NormalTok{,}
    \AttributeTok{caption =} \StringTok{"Figure 3. Relationship between Liquor Retail and Assaults Count"}
\NormalTok{  ) }\SpecialCharTok{+} 
  \FunctionTok{scale\_y\_continuous}\NormalTok{() }\SpecialCharTok{+}
  \FunctionTok{scale\_x\_continuous}\NormalTok{(}\AttributeTok{breaks =}\NormalTok{ scales}\SpecialCharTok{::}\FunctionTok{pretty\_breaks}\NormalTok{(}\DecValTok{5}\NormalTok{)) }\SpecialCharTok{+}
  \FunctionTok{theme}\NormalTok{(}
    \AttributeTok{plot.title =} \FunctionTok{element\_text}\NormalTok{(}\AttributeTok{face =} \StringTok{"bold"}\NormalTok{, }\AttributeTok{hjust =} \FloatTok{0.5}\NormalTok{, }\AttributeTok{size =} \DecValTok{10}\NormalTok{, }\AttributeTok{color =} \StringTok{"darkblue"}\NormalTok{),}
    \AttributeTok{plot.caption =} \FunctionTok{element\_text}\NormalTok{(}\AttributeTok{hjust =} \DecValTok{1}\NormalTok{, }\AttributeTok{face =} \StringTok{"italic"}\NormalTok{, }\AttributeTok{size =} \DecValTok{4}\NormalTok{, }\AttributeTok{color =} \StringTok{"darkgray"}\NormalTok{),}
    \AttributeTok{axis.title.x =} \FunctionTok{element\_text}\NormalTok{(}\AttributeTok{face =} \StringTok{"bold"}\NormalTok{, }\AttributeTok{color =} \StringTok{"darkblue"}\NormalTok{, }\AttributeTok{size =} \DecValTok{6}\NormalTok{),}
    \AttributeTok{axis.title.y =} \FunctionTok{element\_text}\NormalTok{(}\AttributeTok{face =} \StringTok{"bold"}\NormalTok{, }\AttributeTok{color =} \StringTok{"darkblue"}\NormalTok{, }\AttributeTok{size =} \DecValTok{6}\NormalTok{),}
    \AttributeTok{axis.text =} \FunctionTok{element\_text}\NormalTok{(}\AttributeTok{size =} \DecValTok{6}\NormalTok{, }\AttributeTok{color =} \StringTok{"black"}\NormalTok{),}
    \AttributeTok{panel.grid.major =} \FunctionTok{element\_line}\NormalTok{(}\AttributeTok{color =} \StringTok{"lightgray"}\NormalTok{, }\AttributeTok{linetype =} \StringTok{"dashed"}\NormalTok{),}
    \AttributeTok{panel.grid.minor =} \FunctionTok{element\_blank}\NormalTok{(),}
    \AttributeTok{plot.background =} \FunctionTok{element\_rect}\NormalTok{(}\AttributeTok{fill =} \StringTok{"white"}\NormalTok{, }\AttributeTok{color =} \ConstantTok{NA}\NormalTok{)}
\NormalTok{  )}

\CommentTok{\#POD Cameras}
\NormalTok{plot3 }\OtherTok{\textless{}{-}}\FunctionTok{ggplot}\NormalTok{(combined, }\FunctionTok{aes}\NormalTok{(}\AttributeTok{x =} \StringTok{\textasciigrave{}}\AttributeTok{POD.Cameras}\StringTok{\textasciigrave{}}\NormalTok{, }\AttributeTok{y =} \StringTok{\textasciigrave{}}\AttributeTok{countAssaults}\StringTok{\textasciigrave{}}\NormalTok{)) }\SpecialCharTok{+} 
  \FunctionTok{geom\_point}\NormalTok{(}\AttributeTok{color =} \StringTok{"dodgerblue"}\NormalTok{, }\AttributeTok{alpha =} \FloatTok{0.6}\NormalTok{, }\AttributeTok{size =} \DecValTok{3}\NormalTok{) }\SpecialCharTok{+}
  \FunctionTok{geom\_smooth}\NormalTok{(}\AttributeTok{method =} \StringTok{"lm"}\NormalTok{, }\AttributeTok{color =} \StringTok{"red"}\NormalTok{, }\AttributeTok{se =} \ConstantTok{FALSE}\NormalTok{, }\AttributeTok{linetype =} \StringTok{"dashed"}\NormalTok{) }\SpecialCharTok{+}
  \FunctionTok{labs}\NormalTok{(}
    \AttributeTok{title =} \StringTok{"POD Cameras vs Assaults Count"}\NormalTok{,  }
    \AttributeTok{x =} \StringTok{"Number of Cameras"}\NormalTok{,}
    \AttributeTok{y =} \StringTok{"Number of Assaults"}\NormalTok{,}
    \AttributeTok{caption =} \StringTok{"Figure 3. Relationship between POD Camera and Assaults Count"}
\NormalTok{  ) }\SpecialCharTok{+} 
  \FunctionTok{scale\_y\_continuous}\NormalTok{() }\SpecialCharTok{+}
  \FunctionTok{scale\_x\_continuous}\NormalTok{() }\SpecialCharTok{+} 
  \FunctionTok{theme\_minimal}\NormalTok{(}\AttributeTok{base\_size =} \DecValTok{12}\NormalTok{) }\SpecialCharTok{+}
  \FunctionTok{theme}\NormalTok{(}
    \AttributeTok{plot.title =} \FunctionTok{element\_text}\NormalTok{(}\AttributeTok{face =} \StringTok{"bold"}\NormalTok{, }\AttributeTok{hjust =} \FloatTok{0.5}\NormalTok{, }\AttributeTok{size =} \DecValTok{10}\NormalTok{, }\AttributeTok{color =} \StringTok{"darkred"}\NormalTok{),}
    \AttributeTok{plot.caption =} \FunctionTok{element\_text}\NormalTok{(}\AttributeTok{hjust =} \DecValTok{1}\NormalTok{, }\AttributeTok{face =} \StringTok{"italic"}\NormalTok{, }\AttributeTok{size =} \DecValTok{4}\NormalTok{, }\AttributeTok{color =} \StringTok{"darkgray"}\NormalTok{),}
    \AttributeTok{axis.title.x =} \FunctionTok{element\_text}\NormalTok{(}\AttributeTok{face =} \StringTok{"bold"}\NormalTok{, }\AttributeTok{color =} \StringTok{"darkblue"}\NormalTok{, }\AttributeTok{size =} \DecValTok{6}\NormalTok{),}
    \AttributeTok{axis.title.y =} \FunctionTok{element\_text}\NormalTok{(}\AttributeTok{face =} \StringTok{"bold"}\NormalTok{, }\AttributeTok{color =} \StringTok{"darkblue"}\NormalTok{, }\AttributeTok{size =} \DecValTok{6}\NormalTok{),}
    \AttributeTok{axis.text =} \FunctionTok{element\_text}\NormalTok{(}\AttributeTok{size =} \DecValTok{6}\NormalTok{, }\AttributeTok{color =} \StringTok{"black"}\NormalTok{),}
    \AttributeTok{panel.grid.major =} \FunctionTok{element\_line}\NormalTok{(}\AttributeTok{color =} \StringTok{"lightgray"}\NormalTok{, }\AttributeTok{linetype =} \StringTok{"dashed"}\NormalTok{),}
    \AttributeTok{panel.grid.minor =} \FunctionTok{element\_blank}\NormalTok{(),}
    \AttributeTok{plot.background =} \FunctionTok{element\_rect}\NormalTok{(}\AttributeTok{fill =} \StringTok{"white"}\NormalTok{, }\AttributeTok{color =} \ConstantTok{NA}\NormalTok{)}
\NormalTok{  )}

\CommentTok{\#Street Lights Out}
\NormalTok{plot4 }\OtherTok{\textless{}{-}} \FunctionTok{ggplot}\NormalTok{(combined, }\FunctionTok{aes}\NormalTok{(}\AttributeTok{x =} \StringTok{\textasciigrave{}}\AttributeTok{Street\_Lights\_Out}\StringTok{\textasciigrave{}}\NormalTok{, }\AttributeTok{y =} \StringTok{\textasciigrave{}}\AttributeTok{countAssaults}\StringTok{\textasciigrave{}}\NormalTok{)) }\SpecialCharTok{+} 
  \FunctionTok{geom\_point}\NormalTok{(}\AttributeTok{color =} \StringTok{"dodgerblue"}\NormalTok{, }\AttributeTok{alpha =} \FloatTok{0.6}\NormalTok{, }\AttributeTok{size =} \DecValTok{3}\NormalTok{) }\SpecialCharTok{+}
  \FunctionTok{geom\_smooth}\NormalTok{(}\AttributeTok{method =} \StringTok{"lm"}\NormalTok{, }\AttributeTok{color =} \StringTok{"red"}\NormalTok{, }\AttributeTok{se =} \ConstantTok{FALSE}\NormalTok{, }\AttributeTok{linetype =} \StringTok{"dashed"}\NormalTok{) }\SpecialCharTok{+}
  \FunctionTok{labs}\NormalTok{(}
    \AttributeTok{title =} \StringTok{"Street Lights Out vs Assaults Count"}\NormalTok{,  }
    \AttributeTok{x =} \StringTok{"Number of Street Lights Out"}\NormalTok{,}
    \AttributeTok{y =} \StringTok{"Number of Assaults"}\NormalTok{,}
    \AttributeTok{caption =} \StringTok{"Figure 3. Relationship between Street Lights Out and Assaults Count"}
\NormalTok{  ) }\SpecialCharTok{+} 
  \FunctionTok{scale\_y\_continuous}\NormalTok{() }\SpecialCharTok{+}
  \FunctionTok{scale\_x\_continuous}\NormalTok{() }\SpecialCharTok{+} 
  \FunctionTok{theme\_minimal}\NormalTok{(}\AttributeTok{base\_size =} \DecValTok{12}\NormalTok{) }\SpecialCharTok{+}
  \FunctionTok{theme}\NormalTok{(}
    \AttributeTok{plot.title =} \FunctionTok{element\_text}\NormalTok{(}\AttributeTok{face =} \StringTok{"bold"}\NormalTok{, }\AttributeTok{hjust =} \FloatTok{0.5}\NormalTok{, }\AttributeTok{size =} \DecValTok{10}\NormalTok{, }\AttributeTok{color =} \StringTok{"darkgreen"}\NormalTok{),}
    \AttributeTok{plot.caption =} \FunctionTok{element\_text}\NormalTok{(}\AttributeTok{hjust =} \DecValTok{1}\NormalTok{, }\AttributeTok{face =} \StringTok{"italic"}\NormalTok{, }\AttributeTok{size =} \DecValTok{4}\NormalTok{, }\AttributeTok{color =} \StringTok{"darkgray"}\NormalTok{),}
    \AttributeTok{axis.title.x =} \FunctionTok{element\_text}\NormalTok{(}\AttributeTok{face =} \StringTok{"bold"}\NormalTok{, }\AttributeTok{color =} \StringTok{"darkblue"}\NormalTok{, }\AttributeTok{size =} \DecValTok{6}\NormalTok{),}
    \AttributeTok{axis.title.y =} \FunctionTok{element\_text}\NormalTok{(}\AttributeTok{face =} \StringTok{"bold"}\NormalTok{, }\AttributeTok{color =} \StringTok{"darkblue"}\NormalTok{, }\AttributeTok{size =} \DecValTok{6}\NormalTok{),}
    \AttributeTok{axis.text =} \FunctionTok{element\_text}\NormalTok{(}\AttributeTok{size =} \DecValTok{6}\NormalTok{, }\AttributeTok{color =} \StringTok{"black"}\NormalTok{),}
    \AttributeTok{panel.grid.major =} \FunctionTok{element\_line}\NormalTok{(}\AttributeTok{color =} \StringTok{"lightgray"}\NormalTok{, }\AttributeTok{linetype =} \StringTok{"dashed"}\NormalTok{),}
    \AttributeTok{panel.grid.minor =} \FunctionTok{element\_blank}\NormalTok{(),}
    \AttributeTok{plot.background =} \FunctionTok{element\_rect}\NormalTok{(}\AttributeTok{fill =} \StringTok{"white"}\NormalTok{, }\AttributeTok{color =} \ConstantTok{NA}\NormalTok{)}
\NormalTok{  )}

\FunctionTok{grid.arrange}\NormalTok{(plot1, plot2, plot3, plot4, }\AttributeTok{ncol =} \DecValTok{2}\NormalTok{)}
\end{Highlighting}
\end{Shaded}

\includegraphics{Indriaty_TututHW3_files/figure-latex/scatterplots-1.pdf}

\textbf{c.~Feature Engineering}

First, we use nearest neighbor feature to check exposures by distance.
By choosing k=3, the maps will show the \textbf{3 nearest neighboring
points} and use those to calculate the proximity to certain risk
factors.The maps below show four independent variables and the distance
from each grid cell to its nearest variables each. We can see
\textbf{most of the areas for all variables are in the lowest range of
average distance}. For POD Cameras, it shows the least exposures are
only on the most northern and southern part of Chicago. For Liquor
retail and street lights out, the least exposure is in the most southern
part of Chicago. For Metra Stops, the least exposure is when it is far
from metra stops.

\begin{Shaded}
\begin{Highlighting}[]
\NormalTok{st\_c }\OtherTok{\textless{}{-}}\NormalTok{ st\_coordinates}
\NormalTok{st\_coid }\OtherTok{\textless{}{-}}\NormalTok{ st\_centroid}

\NormalTok{vars\_net }\OtherTok{\textless{}{-}}
\NormalTok{  vars\_net }\SpecialCharTok{\%\textgreater{}\%}
    \FunctionTok{mutate}\NormalTok{(}
      \AttributeTok{cctv.nn =}
        \FunctionTok{nn\_function}\NormalTok{(}\FunctionTok{st\_c}\NormalTok{(}\FunctionTok{st\_coid}\NormalTok{(vars\_net)), }\FunctionTok{st\_c}\NormalTok{(cctv),}\DecValTok{3}\NormalTok{),}
      \AttributeTok{Liquor\_Retail.nn =}
        \FunctionTok{nn\_function}\NormalTok{(}\FunctionTok{st\_c}\NormalTok{(}\FunctionTok{st\_coid}\NormalTok{(vars\_net)), }\FunctionTok{st\_c}\NormalTok{(liquorRetail),}\DecValTok{3}\NormalTok{),}
      \AttributeTok{Street\_Lights\_Out.nn =}
        \FunctionTok{nn\_function}\NormalTok{(}\FunctionTok{st\_c}\NormalTok{(}\FunctionTok{st\_coid}\NormalTok{(vars\_net)), }\FunctionTok{st\_c}\NormalTok{(streetLightsOut),}\DecValTok{3}\NormalTok{),}
      \AttributeTok{metraStops.nn =}
        \FunctionTok{nn\_function}\NormalTok{(}\FunctionTok{st\_c}\NormalTok{(}\FunctionTok{st\_coid}\NormalTok{(vars\_net)), }\FunctionTok{st\_c}\NormalTok{(metraStops),}\DecValTok{3}\NormalTok{))}

\NormalTok{vars\_net.long.nn }\OtherTok{\textless{}{-}} 
\NormalTok{  dplyr}\SpecialCharTok{::}\FunctionTok{select}\NormalTok{(vars\_net, }\FunctionTok{ends\_with}\NormalTok{(}\StringTok{".nn"}\NormalTok{)) }\SpecialCharTok{\%\textgreater{}\%}
    \FunctionTok{gather}\NormalTok{(Variable, value, }\SpecialCharTok{{-}}\NormalTok{geometry)}

\NormalTok{vars }\OtherTok{\textless{}{-}} \FunctionTok{unique}\NormalTok{(vars\_net.long.nn}\SpecialCharTok{$}\NormalTok{Variable)}
\NormalTok{mapList }\OtherTok{\textless{}{-}} \FunctionTok{list}\NormalTok{()}

\ControlFlowTok{for}\NormalTok{(i }\ControlFlowTok{in}\NormalTok{ vars)\{}
\NormalTok{  mapList[[i]] }\OtherTok{\textless{}{-}} 
    \FunctionTok{ggplot}\NormalTok{() }\SpecialCharTok{+}
      \FunctionTok{geom\_sf}\NormalTok{(}\AttributeTok{data =} \FunctionTok{filter}\NormalTok{(vars\_net.long.nn, Variable }\SpecialCharTok{==}\NormalTok{ i), }\FunctionTok{aes}\NormalTok{(}\AttributeTok{fill=}\NormalTok{value), }\AttributeTok{colour=}\ConstantTok{NA}\NormalTok{) }\SpecialCharTok{+}
      \FunctionTok{scale\_fill\_viridis}\NormalTok{(}\AttributeTok{name=}\StringTok{""}\NormalTok{) }\SpecialCharTok{+}
      \FunctionTok{labs}\NormalTok{(}\AttributeTok{title=}\NormalTok{i) }\SpecialCharTok{+}
      \FunctionTok{mapTheme}\NormalTok{()\}}

\FunctionTok{do.call}\NormalTok{(grid.arrange,}\FunctionTok{c}\NormalTok{(mapList, }\AttributeTok{ncol =} \DecValTok{2}\NormalTok{, }\AttributeTok{top =} \StringTok{"Nearest Neighbor risk Factors by Fishnet"}\NormalTok{))}
\end{Highlighting}
\end{Shaded}

\includegraphics{Indriaty_TututHW3_files/figure-latex/vars-1.pdf}

Second, we use Euclidean Distance to the loop which means the distance
to the city center of Chicago. The map also shows \textbf{the closer we
are to the city center, the higher the exposure of crime we have.}

\begin{Shaded}
\begin{Highlighting}[]
\NormalTok{loopPoint }\OtherTok{\textless{}{-}}
  \FunctionTok{filter}\NormalTok{(neighborhoods, name }\SpecialCharTok{==} \StringTok{"Loop"}\NormalTok{) }\SpecialCharTok{\%\textgreater{}\%}
  \FunctionTok{st\_centroid}\NormalTok{()}

\NormalTok{vars\_net}\SpecialCharTok{$}\NormalTok{loopDistance }\OtherTok{=}
  \FunctionTok{st\_distance}\NormalTok{(}\FunctionTok{st\_centroid}\NormalTok{(vars\_net),loopPoint) }\SpecialCharTok{\%\textgreater{}\%}
  \FunctionTok{as.numeric}\NormalTok{() }

\FunctionTok{ggplot}\NormalTok{() }\SpecialCharTok{+}
  \FunctionTok{geom\_sf}\NormalTok{(}\AttributeTok{data =}\NormalTok{ vars\_net, }\FunctionTok{aes}\NormalTok{(}\AttributeTok{fill =}\NormalTok{ loopDistance), }\AttributeTok{color =} \ConstantTok{NA}\NormalTok{) }\SpecialCharTok{+}
  \FunctionTok{scale\_fill\_viridis\_c}\NormalTok{(}\AttributeTok{option =} \StringTok{"plasma"}\NormalTok{, }\AttributeTok{name =} \StringTok{"Distance to Loop (m)"}\NormalTok{) }\SpecialCharTok{+}
  \FunctionTok{labs}\NormalTok{(}\AttributeTok{title =} \StringTok{"Euclidean Distance to the Loop"}\NormalTok{, }
       \AttributeTok{caption =} \StringTok{"Distance measured in meters"}\NormalTok{) }\SpecialCharTok{+}
  \FunctionTok{theme\_minimal}\NormalTok{() }\SpecialCharTok{+}
  \FunctionTok{theme}\NormalTok{(}\AttributeTok{legend.position =} \StringTok{"right"}\NormalTok{,}
        \AttributeTok{plot.title =} \FunctionTok{element\_text}\NormalTok{(}\AttributeTok{hjust =} \FloatTok{0.5}\NormalTok{))}
\end{Highlighting}
\end{Shaded}

\includegraphics{Indriaty_TututHW3_files/figure-latex/loop-1.pdf}

Lastly, we use Police District and Neighborhoods data to be joined with
the dependent and independent variables for regression model.

\begin{Shaded}
\begin{Highlighting}[]
\NormalTok{final\_net }\OtherTok{\textless{}{-}}
  \FunctionTok{left\_join}\NormalTok{(crime\_net, }\FunctionTok{st\_drop\_geometry}\NormalTok{(vars\_net), }\AttributeTok{by=}\StringTok{"uniqueID"}\NormalTok{) }

\NormalTok{policeDistricts }\OtherTok{\textless{}{-}} 
  \FunctionTok{st\_read}\NormalTok{(}\StringTok{"https://data.cityofchicago.org/api/geospatial/fthy{-}xz3r?method=export\&format=GeoJSON"}\NormalTok{) }\SpecialCharTok{\%\textgreater{}\%}
  \FunctionTok{st\_transform}\NormalTok{(}\StringTok{\textquotesingle{}ESRI:102271\textquotesingle{}}\NormalTok{) }\SpecialCharTok{\%\textgreater{}\%}
\NormalTok{  dplyr}\SpecialCharTok{::}\FunctionTok{select}\NormalTok{(}\AttributeTok{District =}\NormalTok{ dist\_num)}

\NormalTok{final\_net }\OtherTok{\textless{}{-}}
  \FunctionTok{st\_centroid}\NormalTok{(final\_net) }\SpecialCharTok{\%\textgreater{}\%}
    \FunctionTok{st\_join}\NormalTok{(dplyr}\SpecialCharTok{::}\FunctionTok{select}\NormalTok{(neighborhoods, name)) }\SpecialCharTok{\%\textgreater{}\%}
    \FunctionTok{st\_join}\NormalTok{(dplyr}\SpecialCharTok{::}\FunctionTok{select}\NormalTok{(policeDistricts, District)) }\SpecialCharTok{\%\textgreater{}\%}
      \FunctionTok{st\_drop\_geometry}\NormalTok{() }\SpecialCharTok{\%\textgreater{}\%}
      \FunctionTok{left\_join}\NormalTok{(dplyr}\SpecialCharTok{::}\FunctionTok{select}\NormalTok{(final\_net, geometry, uniqueID)) }\SpecialCharTok{\%\textgreater{}\%}
      \FunctionTok{st\_sf}\NormalTok{() }\SpecialCharTok{\%\textgreater{}\%}
  \FunctionTok{na.omit}\NormalTok{()}

\FunctionTok{grid.arrange}\NormalTok{(}\AttributeTok{ncol=}\DecValTok{2}\NormalTok{,}
\FunctionTok{ggplot}\NormalTok{() }\SpecialCharTok{+}
      \FunctionTok{geom\_sf}\NormalTok{(}\AttributeTok{data =}\NormalTok{ final\_net, }\FunctionTok{aes}\NormalTok{(}\AttributeTok{fill=}\NormalTok{District)) }\SpecialCharTok{+}
       \FunctionTok{scale\_fill\_viridis\_d}\NormalTok{(}\AttributeTok{option =} \StringTok{"plasma"}\NormalTok{) }\SpecialCharTok{+}
      \FunctionTok{labs}\NormalTok{(}\AttributeTok{title=}\StringTok{"Police Districts"}\NormalTok{) }\SpecialCharTok{+}
      \FunctionTok{mapTheme}\NormalTok{()}\SpecialCharTok{+}
      \FunctionTok{theme}\NormalTok{(}\AttributeTok{legend.position =} \StringTok{"none"}\NormalTok{),}

\FunctionTok{ggplot}\NormalTok{() }\SpecialCharTok{+}
      \FunctionTok{geom\_sf}\NormalTok{(}\AttributeTok{data =}\NormalTok{ final\_net, }\FunctionTok{aes}\NormalTok{(}\AttributeTok{fill=}\NormalTok{name)) }\SpecialCharTok{+}
       \FunctionTok{scale\_fill\_viridis\_d}\NormalTok{(}\AttributeTok{option =} \StringTok{"plasma"}\NormalTok{) }\SpecialCharTok{+}
      \FunctionTok{labs}\NormalTok{(}\AttributeTok{title=}\StringTok{"Neighborhoods"}\NormalTok{) }\SpecialCharTok{+}
      \FunctionTok{mapTheme}\NormalTok{()}\SpecialCharTok{+}
      \FunctionTok{theme}\NormalTok{(}\AttributeTok{legend.position =} \StringTok{"none"}\NormalTok{))}
\end{Highlighting}
\end{Shaded}

\includegraphics{Indriaty_TututHW3_files/figure-latex/finalnet-1.pdf}

\section{Local Moran's I}\label{local-morans-i}

We use local moran's I to assess spatial autocorrelation at a local
level and identify clusters or outliers in the spatial distribution of
variables. This detects hotspots (clusters of high values), coldspots
(clusters of low values), and spatial outliers (areas where a location
has a value that is significantly different from its neighbors).

The result for Local Moran's I map above display \textbf{hotspots
(brighter color) meaning high crime rate surrounded by other high-crime
areas, which is in the city center.} For P-Value map and significant
hotspots, we will talk about that further.

This information is useful to realize that a model needs to be
generalizable therefore it can predict equally well in the hotspots and
in the coldspots. We can see from \textbf{assaults count map and
significant hotspots that they align which means a possible good
generalization}.

\begin{Shaded}
\begin{Highlighting}[]
\NormalTok{final\_net.nb }\OtherTok{\textless{}{-}} \FunctionTok{poly2nb}\NormalTok{(}\FunctionTok{as\_Spatial}\NormalTok{(final\_net), }\AttributeTok{queen=}\ConstantTok{TRUE}\NormalTok{)}
\NormalTok{final\_net.weights }\OtherTok{\textless{}{-}} \FunctionTok{nb2listw}\NormalTok{(final\_net.nb, }\AttributeTok{style=}\StringTok{"W"}\NormalTok{, }\AttributeTok{zero.policy=}\ConstantTok{TRUE}\NormalTok{)}

\FunctionTok{print}\NormalTok{(final\_net.weights, }\AttributeTok{zero.policy=}\ConstantTok{TRUE}\NormalTok{)}

\NormalTok{local\_morans }\OtherTok{\textless{}{-}} \FunctionTok{localmoran}\NormalTok{(final\_net}\SpecialCharTok{$}\NormalTok{countAssaults, final\_net.weights, }\AttributeTok{zero.policy=}\ConstantTok{TRUE}\NormalTok{) }\SpecialCharTok{\%\textgreater{}\%}                 \FunctionTok{as.data.frame}\NormalTok{()}

\CommentTok{\# join local Moran\textquotesingle{}s I results to fishnet}
\NormalTok{final\_net.localMorans }\OtherTok{\textless{}{-}} 
  \FunctionTok{cbind}\NormalTok{(local\_morans, }\FunctionTok{as.data.frame}\NormalTok{(final\_net)) }\SpecialCharTok{\%\textgreater{}\%} 
  \FunctionTok{st\_sf}\NormalTok{() }\SpecialCharTok{\%\textgreater{}\%}
\NormalTok{  dplyr}\SpecialCharTok{::}\FunctionTok{select}\NormalTok{(}\AttributeTok{Assaults\_Count =}\NormalTok{ countAssaults, }
                \AttributeTok{Local\_Morans\_I =}\NormalTok{ Ii, }
                \AttributeTok{P\_Value =} \StringTok{\textasciigrave{}}\AttributeTok{Pr(z != E(Ii))}\StringTok{\textasciigrave{}}\NormalTok{) }\SpecialCharTok{\%\textgreater{}\%}
  \FunctionTok{mutate}\NormalTok{(}\AttributeTok{Significant\_Hotspots =} \FunctionTok{ifelse}\NormalTok{(P\_Value }\SpecialCharTok{\textless{}=} \FloatTok{0.001}\NormalTok{, }\DecValTok{1}\NormalTok{, }\DecValTok{0}\NormalTok{)) }\SpecialCharTok{\%\textgreater{}\%}
  \FunctionTok{gather}\NormalTok{(Variable, Value, }\SpecialCharTok{{-}}\NormalTok{geometry)}

\NormalTok{vars }\OtherTok{\textless{}{-}} \FunctionTok{unique}\NormalTok{(final\_net.localMorans}\SpecialCharTok{$}\NormalTok{Variable)}
\NormalTok{varList }\OtherTok{\textless{}{-}} \FunctionTok{list}\NormalTok{()}

\ControlFlowTok{for}\NormalTok{(i }\ControlFlowTok{in}\NormalTok{ vars)\{}
\NormalTok{  varList[[i]] }\OtherTok{\textless{}{-}} 
    \FunctionTok{ggplot}\NormalTok{() }\SpecialCharTok{+}
      \FunctionTok{geom\_sf}\NormalTok{(}\AttributeTok{data =} \FunctionTok{filter}\NormalTok{(final\_net.localMorans, Variable }\SpecialCharTok{==}\NormalTok{ i), }
              \FunctionTok{aes}\NormalTok{(}\AttributeTok{fill =}\NormalTok{ Value), }\AttributeTok{colour=}\ConstantTok{NA}\NormalTok{) }\SpecialCharTok{+}
      \FunctionTok{scale\_fill\_viridis}\NormalTok{(}\AttributeTok{option =} \StringTok{"plasma"}\NormalTok{, }\AttributeTok{name=}\StringTok{""}\NormalTok{) }\SpecialCharTok{+}
      \FunctionTok{labs}\NormalTok{(}\AttributeTok{title=}\NormalTok{i) }\SpecialCharTok{+}
      \FunctionTok{mapTheme}\NormalTok{() }\SpecialCharTok{+} \FunctionTok{theme}\NormalTok{(}\AttributeTok{legend.position=}\StringTok{"bottom"}\NormalTok{)\}}

\FunctionTok{do.call}\NormalTok{(grid.arrange,}\FunctionTok{c}\NormalTok{(varList, }\AttributeTok{ncol =} \DecValTok{4}\NormalTok{, }\AttributeTok{top =} \StringTok{"Local Morans I statistics, Assaults"}\NormalTok{))}
\end{Highlighting}
\end{Shaded}

\includegraphics{Indriaty_TututHW3_files/figure-latex/moransi-1.pdf}

We distribute the significance which is shown by the p-value to 4
different categories. For the p-value nearing 0.1, the significant
hotspots increase. This means that \textbf{more hotspots are
insignificant with the clusters and the observed local Moran's I
statistic is likely due to random chance}.

\begin{Shaded}
\begin{Highlighting}[]
\NormalTok{final\_net.localMoransbypval }\OtherTok{\textless{}{-}} 
  \FunctionTok{cbind}\NormalTok{(local\_morans, }\FunctionTok{as.data.frame}\NormalTok{(final\_net)) }\SpecialCharTok{\%\textgreater{}\%} 
  \FunctionTok{st\_sf}\NormalTok{() }\SpecialCharTok{\%\textgreater{}\%}
\NormalTok{  dplyr}\SpecialCharTok{::}\FunctionTok{select}\NormalTok{(}\AttributeTok{Assaults\_Count =}\NormalTok{ countAssaults, }
                \AttributeTok{Local\_Morans\_I =}\NormalTok{ Ii, }
                \AttributeTok{P\_Value =} \StringTok{\textasciigrave{}}\AttributeTok{Pr(z != E(Ii))}\StringTok{\textasciigrave{}}\NormalTok{) }\SpecialCharTok{\%\textgreater{}\%}
  \FunctionTok{mutate}\NormalTok{(}\AttributeTok{Significant\_Hotspots0.1 =} \FunctionTok{ifelse}\NormalTok{(P\_Value }\SpecialCharTok{\textless{}=} \FloatTok{0.1}\NormalTok{ , }\DecValTok{1}\NormalTok{, }\DecValTok{0}\NormalTok{)) }\SpecialCharTok{\%\textgreater{}\%}
  \FunctionTok{mutate}\NormalTok{(}\AttributeTok{Significant\_Hotspots0.001 =} \FunctionTok{ifelse}\NormalTok{(P\_Value }\SpecialCharTok{\textless{}=} \FloatTok{0.001}\NormalTok{ , }\DecValTok{1}\NormalTok{, }\DecValTok{0}\NormalTok{)) }\SpecialCharTok{\%\textgreater{}\%}  
  \FunctionTok{mutate}\NormalTok{(}\AttributeTok{Significant\_Hotspots0.00001 =} \FunctionTok{ifelse}\NormalTok{(P\_Value }\SpecialCharTok{\textless{}=} \FloatTok{0.00001}\NormalTok{ , }\DecValTok{1}\NormalTok{, }\DecValTok{0}\NormalTok{)) }\SpecialCharTok{\%\textgreater{}\%}  
  \FunctionTok{mutate}\NormalTok{(}\AttributeTok{Significant\_Hotspots0.0000001 =} \FunctionTok{ifelse}\NormalTok{(P\_Value }\SpecialCharTok{\textless{}=} \FloatTok{0.0000001}\NormalTok{, }\DecValTok{1}\NormalTok{, }\DecValTok{0}\NormalTok{)) }\SpecialCharTok{\%\textgreater{}\%}  
  \FunctionTok{gather}\NormalTok{(Variable, Value, }\SpecialCharTok{{-}}\NormalTok{geometry)}

\NormalTok{final\_net.localMoransbypval }\OtherTok{\textless{}{-}} \FunctionTok{subset}\NormalTok{(final\_net.localMoransbypval, }\FunctionTok{grepl}\NormalTok{(}\StringTok{"Significant\_Hotspots"}\NormalTok{, Variable)) }

\NormalTok{custom\_labels }\OtherTok{\textless{}{-}} \FunctionTok{c}\NormalTok{(}\StringTok{"Significant\_Hotspots0.1"} \OtherTok{=} \StringTok{"0.1"}\NormalTok{, }
                   \StringTok{"Significant\_Hotspots0.001"} \OtherTok{=} \StringTok{"0.001"}\NormalTok{,}
                   \StringTok{"Significant\_Hotspots0.00001"} \OtherTok{=} \StringTok{"0.00001"}\NormalTok{,}
                   \StringTok{"Significant\_Hotspots0.0000001"} \OtherTok{=} \StringTok{"0.0000001"}\NormalTok{)}

\FunctionTok{ggplot}\NormalTok{() }\SpecialCharTok{+}
      \FunctionTok{geom\_sf}\NormalTok{(}\AttributeTok{data =}\NormalTok{ final\_net.localMoransbypval, }
              \FunctionTok{aes}\NormalTok{(}\AttributeTok{fill =}\NormalTok{ Value), }\AttributeTok{colour=}\ConstantTok{NA}\NormalTok{) }\SpecialCharTok{+}
      \FunctionTok{scale\_fill\_viridis}\NormalTok{(}\AttributeTok{option =} \StringTok{"plasma"}\NormalTok{, }\AttributeTok{name=}\StringTok{""}\NormalTok{) }\SpecialCharTok{+}
      \FunctionTok{labs}\NormalTok{(}\AttributeTok{title=}\StringTok{"Assaults Hotspots of Varing Significance"}\NormalTok{) }\SpecialCharTok{+}
      \FunctionTok{facet\_wrap}\NormalTok{(}\SpecialCharTok{\textasciitilde{}}\NormalTok{Variable, }\AttributeTok{ncol=}\DecValTok{4}\NormalTok{, }\AttributeTok{labeller =} \FunctionTok{labeller}\NormalTok{(}\AttributeTok{Variable =}\NormalTok{ custom\_labels))}\SpecialCharTok{+}
      \FunctionTok{mapTheme}\NormalTok{(}\AttributeTok{title\_size =} \DecValTok{14}\NormalTok{) }\SpecialCharTok{+} \FunctionTok{theme}\NormalTok{(}\AttributeTok{legend.position=}\StringTok{"none"}\NormalTok{)}
\end{Highlighting}
\end{Shaded}

\includegraphics{Indriaty_TututHW3_files/figure-latex/sig-1.pdf}

The map below shows average nearest neighbor distance from each cell
centroid to its nearest significant cluster. We use only the significant
hotspots with p-value less than 0.001. \textbf{The highest average
distance} is only \textbf{in the most southern and northern part of the
city} which \textbf{aligns} with ithaving \textbf{the least exposure of
crime from nearest neighbor feature}.

\begin{Shaded}
\begin{Highlighting}[]
\NormalTok{final\_net }\OtherTok{\textless{}{-}}\NormalTok{ final\_net }\SpecialCharTok{\%\textgreater{}\%} 
  \FunctionTok{mutate}\NormalTok{(}\AttributeTok{assaultscount.isSig =} 
           \FunctionTok{ifelse}\NormalTok{(local\_morans[,}\DecValTok{5}\NormalTok{] }\SpecialCharTok{\textless{}=} \FloatTok{0.001}\NormalTok{, }\DecValTok{1}\NormalTok{, }\DecValTok{0}\NormalTok{)) }\SpecialCharTok{\%\textgreater{}\%}
  \FunctionTok{mutate}\NormalTok{(}\AttributeTok{assaultscount.isSig.dist =} 
           \FunctionTok{nn\_function}\NormalTok{(}\FunctionTok{st\_coordinates}\NormalTok{(}\FunctionTok{st\_centroid}\NormalTok{(final\_net)),}
                       \FunctionTok{st\_coordinates}\NormalTok{(}\FunctionTok{st\_centroid}\NormalTok{(}\FunctionTok{filter}\NormalTok{(final\_net, }
\NormalTok{                                           assaultscount.isSig }\SpecialCharTok{==} \DecValTok{1}\NormalTok{))), }
                       \AttributeTok{k =} \DecValTok{1}\NormalTok{))}

\FunctionTok{ggplot}\NormalTok{() }\SpecialCharTok{+}
      \FunctionTok{geom\_sf}\NormalTok{(}\AttributeTok{data =}\NormalTok{ final\_net, }\FunctionTok{aes}\NormalTok{(}\AttributeTok{fill=}\NormalTok{assaultscount.isSig.dist), }\AttributeTok{colour=}\ConstantTok{NA}\NormalTok{) }\SpecialCharTok{+}
      \FunctionTok{scale\_fill\_viridis}\NormalTok{(}\AttributeTok{option =} \StringTok{"plasma"}\NormalTok{, }\AttributeTok{name=}\StringTok{"NN Distance"}\NormalTok{) }\SpecialCharTok{+}
      \FunctionTok{labs}\NormalTok{(}\AttributeTok{title=}\StringTok{"Distance to highly significant assaults hotspots"}\NormalTok{) }\SpecialCharTok{+}
      \FunctionTok{mapTheme}\NormalTok{()}
\end{Highlighting}
\end{Shaded}

\includegraphics{Indriaty_TututHW3_files/figure-latex/sigmap-1.pdf}

\section{Modeling and
Cross-Validation}\label{modeling-and-cross-validation}

We use geospatial risk modeling to predict future possible crime and
cross-validation to get the most accurate model.

\textbf{a. Choosing Models}

I choose the model with removing one of the independent variables one by
one and use cross-validation to count Mean Absolute Error (MAE).
\textbf{The lowest the MAE, the better the model}.

The result below shows t\textbf{he lowest MAE is interestingly when we
remove the Liquor Retail variable}. Because the difference is not big
enough, I choose to continue using the model with four independent
variables.

\begin{Shaded}
\begin{Highlighting}[]
\NormalTok{reg.vars.full }\OtherTok{\textless{}{-}} \FunctionTok{c}\NormalTok{(}\StringTok{"cctv.nn"}\NormalTok{, }\StringTok{"Street\_Lights\_Out.nn"}\NormalTok{, }\StringTok{"Liquor\_Retail.nn"}\NormalTok{, }\StringTok{"metraStops.nn"}\NormalTok{, }\StringTok{"loopDistance"}\NormalTok{)}

\NormalTok{reg.vars.no\_cctv }\OtherTok{\textless{}{-}} \FunctionTok{setdiff}\NormalTok{(reg.vars.full, }\StringTok{"cctv.nn"}\NormalTok{)}
\NormalTok{reg.vars.no\_street\_lights }\OtherTok{\textless{}{-}} \FunctionTok{setdiff}\NormalTok{(reg.vars.full, }\StringTok{"Street\_Lights\_Out.nn"}\NormalTok{)}
\NormalTok{reg.vars.no\_liquor }\OtherTok{\textless{}{-}} \FunctionTok{setdiff}\NormalTok{(reg.vars.full, }\StringTok{"Liquor\_Retail.nn"}\NormalTok{)}
\NormalTok{reg.vars.no\_metra }\OtherTok{\textless{}{-}} \FunctionTok{setdiff}\NormalTok{(reg.vars.full, }\StringTok{"metraStops.nn"}\NormalTok{)}

\NormalTok{models }\OtherTok{\textless{}{-}} \FunctionTok{list}\NormalTok{(}
  \StringTok{"Full Model"} \OtherTok{=}\NormalTok{ reg.vars.full,}
  \StringTok{"No CCTV"} \OtherTok{=}\NormalTok{ reg.vars.no\_cctv,}
  \StringTok{"No Street Lights"} \OtherTok{=}\NormalTok{ reg.vars.no\_street\_lights,}
  \StringTok{"No Liquor Retail"} \OtherTok{=}\NormalTok{ reg.vars.no\_liquor,}
  \StringTok{"No Metra Stops"} \OtherTok{=}\NormalTok{ reg.vars.no\_metra}
\NormalTok{)}

\NormalTok{model\_results }\OtherTok{\textless{}{-}} \FunctionTok{lapply}\NormalTok{(}\FunctionTok{names}\NormalTok{(models), }\ControlFlowTok{function}\NormalTok{(model\_name) \{}
\NormalTok{  vars }\OtherTok{\textless{}{-}}\NormalTok{ models[[model\_name]]}
\NormalTok{  model\_cv }\OtherTok{\textless{}{-}} \FunctionTok{crossValidate}\NormalTok{(}
    \AttributeTok{dataset =}\NormalTok{ final\_net,}
    \AttributeTok{id =} \StringTok{"name"}\NormalTok{,}
    \AttributeTok{dependentVariable =} \StringTok{"countAssaults"}\NormalTok{,}
    \AttributeTok{indVariables =}\NormalTok{ vars}
\NormalTok{  ) }\SpecialCharTok{\%\textgreater{}\%}
\NormalTok{    dplyr}\SpecialCharTok{::}\FunctionTok{select}\NormalTok{(}\AttributeTok{cvID =}\NormalTok{ name, countAssaults, Prediction, geometry)}
\NormalTok{  model\_summary }\OtherTok{\textless{}{-}}\NormalTok{ model\_cv }\SpecialCharTok{\%\textgreater{}\%}
    \FunctionTok{mutate}\NormalTok{(}\AttributeTok{Error =}\NormalTok{ Prediction }\SpecialCharTok{{-}}\NormalTok{ countAssaults) }\SpecialCharTok{\%\textgreater{}\%}
    \FunctionTok{summarize}\NormalTok{(}
      \AttributeTok{MAE =} \FunctionTok{mean}\NormalTok{(}\FunctionTok{abs}\NormalTok{(Error), }\AttributeTok{na.rm =} \ConstantTok{TRUE}\NormalTok{)}
\NormalTok{    )}
  \FunctionTok{return}\NormalTok{(}\FunctionTok{data.frame}\NormalTok{(}\AttributeTok{Model =}\NormalTok{ model\_name, }\AttributeTok{MAE =}\NormalTok{ model\_summary}\SpecialCharTok{$}\NormalTok{MAE))}
\NormalTok{\})}

\NormalTok{model\_results\_df }\OtherTok{\textless{}{-}} \FunctionTok{do.call}\NormalTok{(rbind, model\_results)}

\NormalTok{model\_results\_df }\SpecialCharTok{\%\textgreater{}\%}
  \FunctionTok{kable}\NormalTok{(}\AttributeTok{caption =} \StringTok{"Comparison of MAE for Different Models"}\NormalTok{) }\SpecialCharTok{\%\textgreater{}\%}
  \FunctionTok{kable\_styling}\NormalTok{(}\StringTok{"striped"}\NormalTok{, }\AttributeTok{full\_width =}\NormalTok{ F)}
\end{Highlighting}
\end{Shaded}

\textbf{b. Final Model and Accuracy}

Using the final model with four independent variables, loop distance,
and the average distance of significant hotspots, we cross-validate the
model to see the accuracy of the model.

\begin{Shaded}
\begin{Highlighting}[]
\NormalTok{reg.vars }\OtherTok{\textless{}{-}} \FunctionTok{c}\NormalTok{(}\StringTok{"cctv.nn"}\NormalTok{, }\StringTok{"Street\_Lights\_Out.nn"}\NormalTok{, }\StringTok{"Liquor\_Retail.nn"}\NormalTok{,}
              \StringTok{"metraStops.nn"}\NormalTok{, }\StringTok{"loopDistance"}\NormalTok{)}

\NormalTok{reg.ss.vars }\OtherTok{\textless{}{-}} \FunctionTok{c}\NormalTok{(}\StringTok{"cctv.nn"}\NormalTok{, }\StringTok{"Street\_Lights\_Out.nn"}\NormalTok{, }\StringTok{"Liquor\_Retail.nn"}\NormalTok{,}
                 \StringTok{"metraStops.nn"}\NormalTok{, }\StringTok{"loopDistance"}\NormalTok{, }\StringTok{"assaultscount.isSig"}\NormalTok{,}
                 \StringTok{"assaultscount.isSig.dist"}\NormalTok{)}

\NormalTok{reg.spatialCV }\OtherTok{\textless{}{-}} \FunctionTok{crossValidate}\NormalTok{(}
  \AttributeTok{dataset =}\NormalTok{ final\_net,}
  \AttributeTok{id =} \StringTok{"name"}\NormalTok{,}
  \AttributeTok{dependentVariable =} \StringTok{"countAssaults"}\NormalTok{,}
  \AttributeTok{indVariables =}\NormalTok{ reg.vars) }\SpecialCharTok{\%\textgreater{}\%}
\NormalTok{    dplyr}\SpecialCharTok{::}\FunctionTok{select}\NormalTok{(}\AttributeTok{cvID =}\NormalTok{ name, countAssaults, Prediction, geometry)}

\NormalTok{reg.ss.spatialCV }\OtherTok{\textless{}{-}} \FunctionTok{crossValidate}\NormalTok{(}
  \AttributeTok{dataset =}\NormalTok{ final\_net,}
  \AttributeTok{id =} \StringTok{"name"}\NormalTok{,                           }
  \AttributeTok{dependentVariable =} \StringTok{"countAssaults"}\NormalTok{,}
  \AttributeTok{indVariables =}\NormalTok{ reg.ss.vars) }\SpecialCharTok{\%\textgreater{}\%}
\NormalTok{    dplyr}\SpecialCharTok{::}\FunctionTok{select}\NormalTok{(}\AttributeTok{cvID =}\NormalTok{ name, countAssaults, Prediction, geometry)}


\NormalTok{reg.summary }\OtherTok{\textless{}{-}} 
  \FunctionTok{rbind}\NormalTok{(}
    \FunctionTok{mutate}\NormalTok{(reg.spatialCV,    }\AttributeTok{Error =}\NormalTok{ Prediction }\SpecialCharTok{{-}}\NormalTok{ countAssaults,}
                             \AttributeTok{Regression =} \StringTok{"Spatial LOGO{-}CV: Just Risk Factors"}\NormalTok{),}
                             
    \FunctionTok{mutate}\NormalTok{(reg.ss.spatialCV, }\AttributeTok{Error =}\NormalTok{ Prediction }\SpecialCharTok{{-}}\NormalTok{ countAssaults,}
                             \AttributeTok{Regression =} \StringTok{"Spatial LOGO{-}CV: Spatial Process"}\NormalTok{)) }\SpecialCharTok{\%\textgreater{}\%}
    \FunctionTok{st\_sf}\NormalTok{() }
\end{Highlighting}
\end{Shaded}

The errors are shown by Mean Error, Mean Absolute Error, and Standard
Deviation of Mean Absolute Error. The histogram below shows the
distribution of MAE for the cross-validation of just risk factors and
the one with the spatial process. There is a difference between the two
histogram that will show the importance of spatial process in this
model. I will write further by the map and table.

\begin{Shaded}
\begin{Highlighting}[]
\NormalTok{error\_by\_reg\_and\_fold }\OtherTok{\textless{}{-}} 
\NormalTok{  reg.summary }\SpecialCharTok{\%\textgreater{}\%}
    \FunctionTok{group\_by}\NormalTok{(Regression, cvID) }\SpecialCharTok{\%\textgreater{}\%} 
    \FunctionTok{summarize}\NormalTok{(}\AttributeTok{Mean\_Error =} \FunctionTok{mean}\NormalTok{(Prediction }\SpecialCharTok{{-}}\NormalTok{ countAssaults, }\AttributeTok{na.rm =}\NormalTok{ T),}
              \AttributeTok{MAE =} \FunctionTok{mean}\NormalTok{(}\FunctionTok{abs}\NormalTok{(Mean\_Error), }\AttributeTok{na.rm =}\NormalTok{ T),}
              \AttributeTok{SD\_MAE =} \FunctionTok{mean}\NormalTok{(}\FunctionTok{abs}\NormalTok{(Mean\_Error), }\AttributeTok{na.rm =}\NormalTok{ T)) }\SpecialCharTok{\%\textgreater{}\%}
  \FunctionTok{ungroup}\NormalTok{()}

\NormalTok{error\_by\_reg\_and\_fold }\SpecialCharTok{\%\textgreater{}\%}
  \FunctionTok{ggplot}\NormalTok{(}\FunctionTok{aes}\NormalTok{(MAE)) }\SpecialCharTok{+} 
    \FunctionTok{geom\_histogram}\NormalTok{(}\AttributeTok{bins =} \DecValTok{30}\NormalTok{, }\AttributeTok{colour=}\StringTok{"black"}\NormalTok{, }\AttributeTok{fill =} \StringTok{"\#FDE725FF"}\NormalTok{) }\SpecialCharTok{+}
    \FunctionTok{facet\_wrap}\NormalTok{(}\SpecialCharTok{\textasciitilde{}}\NormalTok{Regression) }\SpecialCharTok{+}  
    \FunctionTok{geom\_vline}\NormalTok{(}\AttributeTok{xintercept =} \DecValTok{0}\NormalTok{) }\SpecialCharTok{+} \FunctionTok{scale\_x\_continuous}\NormalTok{(}\AttributeTok{breaks =} \FunctionTok{seq}\NormalTok{(}\DecValTok{0}\NormalTok{, }\DecValTok{8}\NormalTok{, }\AttributeTok{by =} \DecValTok{1}\NormalTok{)) }\SpecialCharTok{+} 
    \FunctionTok{labs}\NormalTok{(}\AttributeTok{title=}\StringTok{"Distribution of MAE"}\NormalTok{, }\AttributeTok{subtitle =} \StringTok{"LOGO{-}CV"}\NormalTok{,}
         \AttributeTok{x=}\StringTok{"Mean Absolute Error"}\NormalTok{, }\AttributeTok{y=}\StringTok{"Count"}\NormalTok{) }\SpecialCharTok{+}
    \FunctionTok{plotTheme}\NormalTok{()}
\end{Highlighting}
\end{Shaded}

\includegraphics{Indriaty_TututHW3_files/figure-latex/errors-1.pdf}

The maps below show l\textbf{ower MAE for the model with spatial process
which means better model} and the spatial feature plays an
\textbf{important role} in the prediction.

\begin{Shaded}
\begin{Highlighting}[]
\NormalTok{error\_by\_reg\_and\_fold }\SpecialCharTok{\%\textgreater{}\%}
  \FunctionTok{filter}\NormalTok{(}\FunctionTok{str\_detect}\NormalTok{(Regression, }\StringTok{"LOGO"}\NormalTok{)) }\SpecialCharTok{\%\textgreater{}\%}
  \FunctionTok{ggplot}\NormalTok{() }\SpecialCharTok{+}
    \FunctionTok{geom\_sf}\NormalTok{(}\FunctionTok{aes}\NormalTok{(}\AttributeTok{fill =}\NormalTok{ MAE)) }\SpecialCharTok{+}
    \FunctionTok{facet\_wrap}\NormalTok{(}\SpecialCharTok{\textasciitilde{}}\NormalTok{Regression) }\SpecialCharTok{+}
    \FunctionTok{scale\_fill\_viridis}\NormalTok{(}\AttributeTok{option =} \StringTok{"plasma"}\NormalTok{) }\SpecialCharTok{+}
    \FunctionTok{labs}\NormalTok{(}\AttributeTok{title =} \StringTok{"Assaults errors by LOGO{-}CV Regression"}\NormalTok{) }\SpecialCharTok{+}
    \FunctionTok{mapTheme}\NormalTok{() }\SpecialCharTok{+} \FunctionTok{theme}\NormalTok{(}\AttributeTok{legend.position=}\StringTok{"bottom"}\NormalTok{)}
\end{Highlighting}
\end{Shaded}

\includegraphics{Indriaty_TututHW3_files/figure-latex/maperror-1.pdf}

The table below also shows the mean of MAE and standard deviation of MAE
for the model including spatial process are lower than the model with
just risk factors. The \textbf{lower standard deviation} shows that the
latter model is \textbf{more stable and reliable for predicting} despite
any neighborhood conditions.

\begin{Shaded}
\begin{Highlighting}[]
\FunctionTok{st\_drop\_geometry}\NormalTok{(error\_by\_reg\_and\_fold) }\SpecialCharTok{\%\textgreater{}\%}
  \FunctionTok{group\_by}\NormalTok{(Regression) }\SpecialCharTok{\%\textgreater{}\%} 
    \FunctionTok{summarize}\NormalTok{(}\AttributeTok{Mean\_MAE =} \FunctionTok{round}\NormalTok{(}\FunctionTok{mean}\NormalTok{(MAE), }\DecValTok{2}\NormalTok{),}
              \AttributeTok{SD\_MAE =} \FunctionTok{round}\NormalTok{(}\FunctionTok{sd}\NormalTok{(MAE), }\DecValTok{2}\NormalTok{)) }\SpecialCharTok{\%\textgreater{}\%}
  \FunctionTok{kable}\NormalTok{() }\SpecialCharTok{\%\textgreater{}\%}
    \FunctionTok{kable\_styling}\NormalTok{(}\StringTok{"striped"}\NormalTok{, }\AttributeTok{full\_width =}\NormalTok{ F)}
\end{Highlighting}
\end{Shaded}

\textbf{c.~Neighborhood Racial Context}

Using ACS 5-year estimate data to get racial data of the neighborhood,
we do regression with the two models from before. Shown in the table
below, the mean errors in both regressions for non-white are minus which
mean \textbf{the model is underestimating the number of assaults
occurring in areas with non-white} majority. The model's predictions are
generally lower than the actual observed values of assaults in these
neighborhoods. This result shows another \textbf{bias} in the model.

\begin{Shaded}
\begin{Highlighting}[]
\NormalTok{tracts18 }\OtherTok{\textless{}{-}} 
  \FunctionTok{get\_acs}\NormalTok{(}\AttributeTok{geography =} \StringTok{"tract"}\NormalTok{, }\AttributeTok{variables =} \FunctionTok{c}\NormalTok{(}\StringTok{"B01001\_001E"}\NormalTok{,}\StringTok{"B01001A\_001E"}\NormalTok{), }
          \AttributeTok{year =} \DecValTok{2018}\NormalTok{, }\AttributeTok{state=}\DecValTok{17}\NormalTok{, }\AttributeTok{county=}\DecValTok{031}\NormalTok{, }\AttributeTok{geometry=}\NormalTok{T) }\SpecialCharTok{\%\textgreater{}\%}
  \FunctionTok{st\_transform}\NormalTok{(}\StringTok{\textquotesingle{}ESRI:102271\textquotesingle{}}\NormalTok{)  }\SpecialCharTok{\%\textgreater{}\%} 
\NormalTok{  dplyr}\SpecialCharTok{::}\FunctionTok{select}\NormalTok{(variable, estimate, GEOID) }\SpecialCharTok{\%\textgreater{}\%}
  \FunctionTok{spread}\NormalTok{(variable, estimate) }\SpecialCharTok{\%\textgreater{}\%}
  \FunctionTok{rename}\NormalTok{(}\AttributeTok{TotalPop =}\NormalTok{ B01001\_001,}
         \AttributeTok{NumberWhites =}\NormalTok{ B01001A\_001) }\SpecialCharTok{\%\textgreater{}\%}
  \FunctionTok{mutate}\NormalTok{(}\AttributeTok{percentWhite =}\NormalTok{ NumberWhites }\SpecialCharTok{/}\NormalTok{ TotalPop,}
         \AttributeTok{raceContext =} \FunctionTok{ifelse}\NormalTok{(percentWhite }\SpecialCharTok{\textgreater{}}\NormalTok{ .}\DecValTok{5}\NormalTok{, }\StringTok{"Majority\_White"}\NormalTok{, }\StringTok{"Majority\_Non\_White"}\NormalTok{)) }\SpecialCharTok{\%\textgreater{}\%}
\NormalTok{  .[neighborhoods,]}

\NormalTok{reg.summary }\SpecialCharTok{\%\textgreater{}\%} 
  \FunctionTok{filter}\NormalTok{(}\FunctionTok{str\_detect}\NormalTok{(Regression, }\StringTok{"LOGO"}\NormalTok{)) }\SpecialCharTok{\%\textgreater{}\%}
    \FunctionTok{st\_centroid}\NormalTok{() }\SpecialCharTok{\%\textgreater{}\%}
    \FunctionTok{st\_join}\NormalTok{(tracts18) }\SpecialCharTok{\%\textgreater{}\%}
    \FunctionTok{na.omit}\NormalTok{() }\SpecialCharTok{\%\textgreater{}\%}
      \FunctionTok{st\_drop\_geometry}\NormalTok{() }\SpecialCharTok{\%\textgreater{}\%}
      \FunctionTok{group\_by}\NormalTok{(Regression, raceContext) }\SpecialCharTok{\%\textgreater{}\%}
      \FunctionTok{summarize}\NormalTok{(}\AttributeTok{mean.Error =} \FunctionTok{mean}\NormalTok{(Error, }\AttributeTok{na.rm =}\NormalTok{ T)) }\SpecialCharTok{\%\textgreater{}\%}
      \FunctionTok{spread}\NormalTok{(raceContext, mean.Error) }\SpecialCharTok{\%\textgreater{}\%}
      \FunctionTok{kable}\NormalTok{(}\AttributeTok{caption =} \StringTok{"Mean Error by neighborhood racial context"}\NormalTok{) }\SpecialCharTok{\%\textgreater{}\%}
        \FunctionTok{kable\_styling}\NormalTok{(}\StringTok{"striped"}\NormalTok{, }\AttributeTok{full\_width =}\NormalTok{ F) }
\end{Highlighting}
\end{Shaded}

\section{Kernel's Density and Model
Prediction}\label{kernels-density-and-model-prediction}

\textbf{a. Kernel's Density}

Kernel's Density is another tool to show distribution of data. In this
case, we use three different search radius (1000 ft, 1500 ft, and 2000
ft) to see the density of the number of assaults in the city of Chicago.

The maps below show more clustered hotspots within 1000 ft and less
clustered hotspots within 2000 ft. These maps also show spatial pattern
with a smaller radius highlighting localized hotspots, while a larger
radius highlighting broader patterns and trends. This can help make
decisions about \textbf{resource allocation and intervention
strategies}.

\begin{Shaded}
\begin{Highlighting}[]
\NormalTok{assa\_ppp }\OtherTok{\textless{}{-}} \FunctionTok{as.ppp}\NormalTok{(}\FunctionTok{st\_coordinates}\NormalTok{(assaults\_clipped), }\AttributeTok{W =} \FunctionTok{st\_bbox}\NormalTok{(final\_net))}
\NormalTok{assa\_KD}\FloatTok{.1000} \OtherTok{\textless{}{-}}\NormalTok{ spatstat.explore}\SpecialCharTok{::}\FunctionTok{density.ppp}\NormalTok{(assa\_ppp, }\DecValTok{1000}\NormalTok{)}
\NormalTok{assa\_KD}\FloatTok{.1500} \OtherTok{\textless{}{-}}\NormalTok{ spatstat.explore}\SpecialCharTok{::}\FunctionTok{density.ppp}\NormalTok{(assa\_ppp, }\DecValTok{1500}\NormalTok{)}
\NormalTok{assa\_KD}\FloatTok{.2000} \OtherTok{\textless{}{-}}\NormalTok{ spatstat.explore}\SpecialCharTok{::}\FunctionTok{density.ppp}\NormalTok{(assa\_ppp, }\DecValTok{2000}\NormalTok{)}
\NormalTok{assa\_KD.df }\OtherTok{\textless{}{-}} \FunctionTok{rbind}\NormalTok{(}
  \FunctionTok{mutate}\NormalTok{(}\FunctionTok{data.frame}\NormalTok{(}\FunctionTok{rasterToPoints}\NormalTok{(}\FunctionTok{mask}\NormalTok{(}\FunctionTok{raster}\NormalTok{(assa\_KD}\FloatTok{.1000}\NormalTok{), }\FunctionTok{as}\NormalTok{(neighborhoods, }\StringTok{\textquotesingle{}Spatial\textquotesingle{}}\NormalTok{)))), }\AttributeTok{Legend =} \StringTok{"1000 Ft."}\NormalTok{),}
  \FunctionTok{mutate}\NormalTok{(}\FunctionTok{data.frame}\NormalTok{(}\FunctionTok{rasterToPoints}\NormalTok{(}\FunctionTok{mask}\NormalTok{(}\FunctionTok{raster}\NormalTok{(assa\_KD}\FloatTok{.1500}\NormalTok{), }\FunctionTok{as}\NormalTok{(neighborhoods, }\StringTok{\textquotesingle{}Spatial\textquotesingle{}}\NormalTok{)))), }\AttributeTok{Legend =} \StringTok{"1500 Ft."}\NormalTok{),}
  \FunctionTok{mutate}\NormalTok{(}\FunctionTok{data.frame}\NormalTok{(}\FunctionTok{rasterToPoints}\NormalTok{(}\FunctionTok{mask}\NormalTok{(}\FunctionTok{raster}\NormalTok{(assa\_KD}\FloatTok{.2000}\NormalTok{), }\FunctionTok{as}\NormalTok{(neighborhoods, }\StringTok{\textquotesingle{}Spatial\textquotesingle{}}\NormalTok{)))), }\AttributeTok{Legend =} \StringTok{"2000 Ft."}\NormalTok{)) }

\NormalTok{assa\_KD.df}\SpecialCharTok{$}\NormalTok{Legend }\OtherTok{\textless{}{-}} \FunctionTok{factor}\NormalTok{(assa\_KD.df}\SpecialCharTok{$}\NormalTok{Legend, }\AttributeTok{levels =} \FunctionTok{c}\NormalTok{(}\StringTok{"1000 Ft."}\NormalTok{, }\StringTok{"1500 Ft."}\NormalTok{, }\StringTok{"2000 Ft."}\NormalTok{))}

\FunctionTok{ggplot}\NormalTok{(}\AttributeTok{data=}\NormalTok{assa\_KD.df, }\FunctionTok{aes}\NormalTok{(}\AttributeTok{x=}\NormalTok{x, }\AttributeTok{y=}\NormalTok{y)) }\SpecialCharTok{+}
  \FunctionTok{geom\_raster}\NormalTok{(}\FunctionTok{aes}\NormalTok{(}\AttributeTok{fill=}\NormalTok{layer)) }\SpecialCharTok{+} 
  \FunctionTok{facet\_wrap}\NormalTok{(}\SpecialCharTok{\textasciitilde{}}\NormalTok{Legend) }\SpecialCharTok{+}
  \FunctionTok{coord\_sf}\NormalTok{(}\AttributeTok{crs=}\FunctionTok{st\_crs}\NormalTok{(final\_net)) }\SpecialCharTok{+} 
  \FunctionTok{scale\_fill\_viridis}\NormalTok{(}\AttributeTok{name=}\StringTok{"Density"}\NormalTok{) }\SpecialCharTok{+}
  \FunctionTok{labs}\NormalTok{(}\AttributeTok{title =} \StringTok{"Kernel density with 3 different search radius"}\NormalTok{) }\SpecialCharTok{+}
  \FunctionTok{mapTheme}\NormalTok{(}\AttributeTok{title\_size =} \DecValTok{14}\NormalTok{)}
\end{Highlighting}
\end{Shaded}

\includegraphics{Indriaty_TututHW3_files/figure-latex/KD-1.pdf}

This map shows the density of assaults in 2017 in 1000 ft radius. The
\textbf{hotspots} \textbf{align} with the number of assaults in the
areas, such as \textbf{city center, west Chicago, and South Chicago}.

\begin{Shaded}
\begin{Highlighting}[]
\FunctionTok{as.data.frame}\NormalTok{(assa\_KD}\FloatTok{.1000}\NormalTok{) }\SpecialCharTok{\%\textgreater{}\%}
  \FunctionTok{st\_as\_sf}\NormalTok{(}\AttributeTok{coords =} \FunctionTok{c}\NormalTok{(}\StringTok{"x"}\NormalTok{, }\StringTok{"y"}\NormalTok{), }\AttributeTok{crs =} \FunctionTok{st\_crs}\NormalTok{(final\_net)) }\SpecialCharTok{\%\textgreater{}\%}
  \FunctionTok{aggregate}\NormalTok{(., final\_net, mean) }\SpecialCharTok{\%\textgreater{}\%}
   \FunctionTok{ggplot}\NormalTok{() }\SpecialCharTok{+}
     \FunctionTok{geom\_sf}\NormalTok{(}\FunctionTok{aes}\NormalTok{(}\AttributeTok{fill=}\NormalTok{value)) }\SpecialCharTok{+}
     \FunctionTok{geom\_sf}\NormalTok{(}\AttributeTok{data =} \FunctionTok{sample\_n}\NormalTok{(assaults\_clipped, }\DecValTok{1500}\NormalTok{), }\AttributeTok{size =}\NormalTok{ .}\DecValTok{5}\NormalTok{) }\SpecialCharTok{+}
     \FunctionTok{scale\_fill\_viridis}\NormalTok{(}\AttributeTok{name =} \StringTok{"Density"}\NormalTok{) }\SpecialCharTok{+}
     \FunctionTok{labs}\NormalTok{(}\AttributeTok{title =} \StringTok{"Kernel density of 2017 Assaults"}\NormalTok{) }\SpecialCharTok{+}
     \FunctionTok{mapTheme}\NormalTok{(}\AttributeTok{title\_size =} \DecValTok{14}\NormalTok{)}
\end{Highlighting}
\end{Shaded}

\includegraphics{Indriaty_TututHW3_files/figure-latex/mapkd-1.pdf}

\textbf{b. Comparison with Model Prediction}

Now, we compare the 2018 assaults data with the Model Prediction using
2017 assaults data. Overall, \textbf{the highest risks (5th category)}
are still in the three areas mentioned before: \textbf{the city center,
west Chicago, and south Chicago}. The maps below show while there are
similarities in a few areas, there are differences in many more areas.

While the 2018 assaults show growing areas of higher risk, model
prediction tends to show reduced highest risk areas.

\begin{Shaded}
\begin{Highlighting}[]
\NormalTok{assaults18 }\OtherTok{\textless{}{-}} 
  \FunctionTok{read.socrata}\NormalTok{(}\StringTok{"https://data.cityofchicago.org/Public{-}Safety/Crimes{-}2018/3i3m{-}jwuy"}\NormalTok{) }\SpecialCharTok{\%\textgreater{}\%} 
  \FunctionTok{filter}\NormalTok{(Primary.Type }\SpecialCharTok{==} \StringTok{"ASSAULT"} \SpecialCharTok{\&} 
\NormalTok{         Description }\SpecialCharTok{==} \StringTok{"SIMPLE"}\NormalTok{) }\SpecialCharTok{\%\textgreater{}\%}
  \FunctionTok{mutate}\NormalTok{(}\AttributeTok{x =} \FunctionTok{gsub}\NormalTok{(}\StringTok{"[()]"}\NormalTok{, }\StringTok{""}\NormalTok{, Location)) }\SpecialCharTok{\%\textgreater{}\%}
  \FunctionTok{separate}\NormalTok{(x,}\AttributeTok{into=} \FunctionTok{c}\NormalTok{(}\StringTok{"Y"}\NormalTok{,}\StringTok{"X"}\NormalTok{), }\AttributeTok{sep=}\StringTok{","}\NormalTok{) }\SpecialCharTok{\%\textgreater{}\%}
  \FunctionTok{mutate}\NormalTok{(}\AttributeTok{X =} \FunctionTok{as.numeric}\NormalTok{(X),}
         \AttributeTok{Y =} \FunctionTok{as.numeric}\NormalTok{(Y)) }\SpecialCharTok{\%\textgreater{}\%} 
\NormalTok{  na.omit }\SpecialCharTok{\%\textgreater{}\%}
  \FunctionTok{st\_as\_sf}\NormalTok{(}\AttributeTok{coords =} \FunctionTok{c}\NormalTok{(}\StringTok{"X"}\NormalTok{, }\StringTok{"Y"}\NormalTok{), }\AttributeTok{crs =} \DecValTok{4326}\NormalTok{, }\AttributeTok{agr =} \StringTok{"constant"}\NormalTok{) }\SpecialCharTok{\%\textgreater{}\%}
  \FunctionTok{st\_transform}\NormalTok{(}\StringTok{\textquotesingle{}ESRI:102271\textquotesingle{}}\NormalTok{) }\SpecialCharTok{\%\textgreater{}\%} 
  \FunctionTok{distinct}\NormalTok{() }\SpecialCharTok{\%\textgreater{}\%}
\NormalTok{  .[fishnet,]}

\NormalTok{assa\_KDE\_sum }\OtherTok{\textless{}{-}} \FunctionTok{as.data.frame}\NormalTok{(assa\_KD}\FloatTok{.1000}\NormalTok{) }\SpecialCharTok{\%\textgreater{}\%}
  \FunctionTok{st\_as\_sf}\NormalTok{(}\AttributeTok{coords =} \FunctionTok{c}\NormalTok{(}\StringTok{"x"}\NormalTok{, }\StringTok{"y"}\NormalTok{), }\AttributeTok{crs =} \FunctionTok{st\_crs}\NormalTok{(final\_net)) }\SpecialCharTok{\%\textgreater{}\%}
  \FunctionTok{aggregate}\NormalTok{(., final\_net, mean) }


\NormalTok{kde\_breaks }\OtherTok{\textless{}{-}} \FunctionTok{classIntervals}\NormalTok{(assa\_KDE\_sum}\SpecialCharTok{$}\NormalTok{value, }
                             \AttributeTok{n =} \DecValTok{5}\NormalTok{, }\StringTok{"fisher"}\NormalTok{)}


\NormalTok{assa\_KDE\_sf }\OtherTok{\textless{}{-}}\NormalTok{ assa\_KDE\_sum }\SpecialCharTok{\%\textgreater{}\%}
  \FunctionTok{mutate}\NormalTok{(}\AttributeTok{label =} \StringTok{"Kernel Density"}\NormalTok{,}
         \AttributeTok{Risk\_Category =}\NormalTok{ classInt}\SpecialCharTok{::}\FunctionTok{findCols}\NormalTok{(kde\_breaks),}
         \AttributeTok{Risk\_Category =} \FunctionTok{case\_when}\NormalTok{(}
\NormalTok{           Risk\_Category }\SpecialCharTok{==} \DecValTok{5} \SpecialCharTok{\textasciitilde{}} \StringTok{"5th"}\NormalTok{,}
\NormalTok{           Risk\_Category }\SpecialCharTok{==} \DecValTok{4} \SpecialCharTok{\textasciitilde{}} \StringTok{"4th"}\NormalTok{,}
\NormalTok{           Risk\_Category }\SpecialCharTok{==} \DecValTok{3} \SpecialCharTok{\textasciitilde{}} \StringTok{"3rd"}\NormalTok{,}
\NormalTok{           Risk\_Category }\SpecialCharTok{==} \DecValTok{2} \SpecialCharTok{\textasciitilde{}} \StringTok{"2nd"}\NormalTok{,}
\NormalTok{           Risk\_Category }\SpecialCharTok{==} \DecValTok{1} \SpecialCharTok{\textasciitilde{}} \StringTok{"1st"}\NormalTok{)) }\SpecialCharTok{\%\textgreater{}\%}
  \FunctionTok{cbind}\NormalTok{(}
    \FunctionTok{aggregate}\NormalTok{(}
\NormalTok{      dplyr}\SpecialCharTok{::}\FunctionTok{select}\NormalTok{(assaults18) }\SpecialCharTok{\%\textgreater{}\%} \FunctionTok{mutate}\NormalTok{(}\AttributeTok{assaCount =} \DecValTok{1}\NormalTok{), ., sum) }\SpecialCharTok{\%\textgreater{}\%}
    \FunctionTok{mutate}\NormalTok{(}\AttributeTok{assaCount =} \FunctionTok{replace\_na}\NormalTok{(assaCount, }\DecValTok{0}\NormalTok{))) }\SpecialCharTok{\%\textgreater{}\%}
\NormalTok{  dplyr}\SpecialCharTok{::}\FunctionTok{select}\NormalTok{(label, Risk\_Category, assaCount)}

\NormalTok{ml\_breaks }\OtherTok{\textless{}{-}} \FunctionTok{classIntervals}\NormalTok{(reg.ss.spatialCV}\SpecialCharTok{$}\NormalTok{Prediction, }
                             \AttributeTok{n =} \DecValTok{5}\NormalTok{, }\StringTok{"fisher"}\NormalTok{)}
\NormalTok{assa\_risk\_sf }\OtherTok{\textless{}{-}}
\NormalTok{  reg.ss.spatialCV }\SpecialCharTok{\%\textgreater{}\%}
  \FunctionTok{mutate}\NormalTok{(}\AttributeTok{label =} \StringTok{"Risk Predictions"}\NormalTok{,}
         \AttributeTok{Risk\_Category =}\NormalTok{classInt}\SpecialCharTok{::}\FunctionTok{findCols}\NormalTok{(ml\_breaks),}
         \AttributeTok{Risk\_Category =} \FunctionTok{case\_when}\NormalTok{(}
\NormalTok{           Risk\_Category }\SpecialCharTok{==} \DecValTok{5} \SpecialCharTok{\textasciitilde{}} \StringTok{"5th"}\NormalTok{,}
\NormalTok{           Risk\_Category }\SpecialCharTok{==} \DecValTok{4} \SpecialCharTok{\textasciitilde{}} \StringTok{"4th"}\NormalTok{,}
\NormalTok{           Risk\_Category }\SpecialCharTok{==} \DecValTok{3} \SpecialCharTok{\textasciitilde{}} \StringTok{"3rd"}\NormalTok{,}
\NormalTok{           Risk\_Category }\SpecialCharTok{==} \DecValTok{2} \SpecialCharTok{\textasciitilde{}} \StringTok{"2nd"}\NormalTok{,}
\NormalTok{           Risk\_Category }\SpecialCharTok{==} \DecValTok{1} \SpecialCharTok{\textasciitilde{}} \StringTok{"1st"}\NormalTok{)) }\SpecialCharTok{\%\textgreater{}\%}
  \FunctionTok{cbind}\NormalTok{(}
    \FunctionTok{aggregate}\NormalTok{(}
\NormalTok{      dplyr}\SpecialCharTok{::}\FunctionTok{select}\NormalTok{(assaults18) }\SpecialCharTok{\%\textgreater{}\%} \FunctionTok{mutate}\NormalTok{(}\AttributeTok{assaCount =} \DecValTok{1}\NormalTok{), ., sum) }\SpecialCharTok{\%\textgreater{}\%}
      \FunctionTok{mutate}\NormalTok{(}\AttributeTok{assaCount =} \FunctionTok{replace\_na}\NormalTok{(assaCount, }\DecValTok{0}\NormalTok{))) }\SpecialCharTok{\%\textgreater{}\%}
\NormalTok{  dplyr}\SpecialCharTok{::}\FunctionTok{select}\NormalTok{(label,Risk\_Category, assaCount)}

\FunctionTok{rbind}\NormalTok{(assa\_KDE\_sf, assa\_risk\_sf) }\SpecialCharTok{\%\textgreater{}\%}
  \FunctionTok{na.omit}\NormalTok{() }\SpecialCharTok{\%\textgreater{}\%}
  \FunctionTok{gather}\NormalTok{(Variable, Value, }\SpecialCharTok{{-}}\NormalTok{label, }\SpecialCharTok{{-}}\NormalTok{Risk\_Category, }\SpecialCharTok{{-}}\NormalTok{geometry) }\SpecialCharTok{\%\textgreater{}\%}
  \FunctionTok{ggplot}\NormalTok{() }\SpecialCharTok{+}
    \FunctionTok{geom\_sf}\NormalTok{(}\FunctionTok{aes}\NormalTok{(}\AttributeTok{fill =}\NormalTok{ Risk\_Category), }\AttributeTok{colour =} \ConstantTok{NA}\NormalTok{) }\SpecialCharTok{+}
    \FunctionTok{geom\_sf}\NormalTok{(}\AttributeTok{data =} \FunctionTok{sample\_n}\NormalTok{(assaults18, }\DecValTok{3000}\NormalTok{), }\AttributeTok{size =}\NormalTok{ .}\DecValTok{5}\NormalTok{, }\AttributeTok{colour =} \StringTok{"black"}\NormalTok{) }\SpecialCharTok{+}
    \FunctionTok{facet\_wrap}\NormalTok{(}\SpecialCharTok{\textasciitilde{}}\NormalTok{label, ) }\SpecialCharTok{+}
    \FunctionTok{scale\_fill\_viridis}\NormalTok{(}\AttributeTok{option =} \StringTok{"plasma"}\NormalTok{, }\AttributeTok{discrete =} \ConstantTok{TRUE}\NormalTok{) }\SpecialCharTok{+}
    \FunctionTok{labs}\NormalTok{(}\AttributeTok{title=}\StringTok{"Comparison of Kernel Density and Risk Predictions"}\NormalTok{,}
         \AttributeTok{subtitle=}\StringTok{"2018 assaults; 2017 assaults risk predictions"}\NormalTok{) }\SpecialCharTok{+}
    \FunctionTok{mapTheme}\NormalTok{(}\AttributeTok{title\_size =} \DecValTok{14}\NormalTok{)}
\end{Highlighting}
\end{Shaded}

\includegraphics{Indriaty_TututHW3_files/figure-latex/2018assa-1.pdf}

The graph below shows that \textbf{the prediction is not very useful for
the highest and second-highest risk because it fails to predict} that
the assaults will be higher rather than lower. The failure to predict
higher assaults in these areas means a need for refining the model.

\begin{Shaded}
\begin{Highlighting}[]
\FunctionTok{rbind}\NormalTok{(assa\_KDE\_sf, assa\_risk\_sf) }\SpecialCharTok{\%\textgreater{}\%}
  \FunctionTok{st\_drop\_geometry}\NormalTok{() }\SpecialCharTok{\%\textgreater{}\%}
  \FunctionTok{na.omit}\NormalTok{() }\SpecialCharTok{\%\textgreater{}\%}
  \FunctionTok{gather}\NormalTok{(Variable, Value, }\SpecialCharTok{{-}}\NormalTok{label, }\SpecialCharTok{{-}}\NormalTok{Risk\_Category) }\SpecialCharTok{\%\textgreater{}\%}
  \FunctionTok{group\_by}\NormalTok{(label, Risk\_Category) }\SpecialCharTok{\%\textgreater{}\%}
  \FunctionTok{summarize}\NormalTok{(}\AttributeTok{countAssaults =} \FunctionTok{sum}\NormalTok{(Value)) }\SpecialCharTok{\%\textgreater{}\%}
  \FunctionTok{ungroup}\NormalTok{() }\SpecialCharTok{\%\textgreater{}\%}
  \FunctionTok{group\_by}\NormalTok{(label) }\SpecialCharTok{\%\textgreater{}\%}
  \FunctionTok{mutate}\NormalTok{(}\AttributeTok{Pcnt\_of\_test\_set\_crimes =}\NormalTok{ countAssaults }\SpecialCharTok{/} \FunctionTok{sum}\NormalTok{(countAssaults)) }\SpecialCharTok{\%\textgreater{}\%}
    \FunctionTok{ggplot}\NormalTok{(}\FunctionTok{aes}\NormalTok{(Risk\_Category,Pcnt\_of\_test\_set\_crimes)) }\SpecialCharTok{+}
      \FunctionTok{geom\_bar}\NormalTok{(}\FunctionTok{aes}\NormalTok{(}\AttributeTok{fill=}\NormalTok{label), }\AttributeTok{position=}\StringTok{"dodge"}\NormalTok{, }\AttributeTok{stat=}\StringTok{"identity"}\NormalTok{) }\SpecialCharTok{+}
      \FunctionTok{scale\_fill\_viridis}\NormalTok{(}\AttributeTok{option =} \StringTok{"plasma"}\NormalTok{, }\AttributeTok{discrete =} \ConstantTok{TRUE}\NormalTok{, }\AttributeTok{name =} \StringTok{"Model"}\NormalTok{) }\SpecialCharTok{+}
      \FunctionTok{labs}\NormalTok{(}\AttributeTok{title =} \StringTok{"Risk prediction vs. Kernel density, 2018 assaults"}\NormalTok{,}
           \AttributeTok{y =} \StringTok{"\% of Test Set Assaults (per model)"}\NormalTok{,}
           \AttributeTok{x =} \StringTok{"Risk Category"}\NormalTok{) }\SpecialCharTok{+}
  \FunctionTok{theme\_bw}\NormalTok{() }\SpecialCharTok{+}
      \FunctionTok{theme}\NormalTok{(}\AttributeTok{axis.text.x =} \FunctionTok{element\_text}\NormalTok{(}\AttributeTok{angle =} \DecValTok{45}\NormalTok{, }\AttributeTok{vjust =} \FloatTok{0.5}\NormalTok{))}
\end{Highlighting}
\end{Shaded}

\includegraphics{Indriaty_TututHW3_files/figure-latex/kdgraph-1.pdf}

\section{Conclusion}\label{conclusion}

To conclude, I \textbf{do not recommend using the algorithm as it is}
because it appears to fail predicting the increase assaults in the
highest and second-highest risk areas in Chicago City. But, I
\textbf{recommend using the modeling process.} The room for improving
the model is still possible with adding or stronger-correlation
independent variables or changing the existing ones here. Even though
along the way, I \textbf{find some reliability and good generalization
in some variables}. This still is not enough to create a useful
prediction modeling.

\textbf{Simple Assaults} as the chosen outcomes also is not an easy
category of crimes to predict because of the \textbf{quite high level of
bias in data}. But this still is important because this assault type is
the \textbf{highest crime reports} in Chicago. \textbf{Better and
stronger variables} need to be included in the modeling to create
reliable prediction. Some variable ideas to add to probably improve the
modeling in the future are religious locations, recreational places,
education level, and income level.

\end{document}
